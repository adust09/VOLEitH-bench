\section{前提知識と関連研究}
本章では、後続のベンチマークと考察を理解するために必要な暗号学的前提を簡潔に整理する。
\subsection{本研究の背景と目的}
近年、ブロックチェーン技術の発展に伴い、そのスケーラビリティとプライバシー保護が重要な課題となっている。ゼロ知識証明(ZKP)は、これらの課題に対する有効な解決策として注目されており、特にブロックチェーン上での計算負荷を軽減し、プライベートな情報を守りながら取引を検証する手段として期待されている。

しかし、多くの既存のZKPプロトコルは、公開検証可能性(Public Verifiability)を達成するために、複雑なTrusted Setupや高コストなペアリング計算を必要とする場合があった。このような背景の中、VOLE-in-the-Head (VOLEitH) プロトコルは、指定検証者証明の課題を克服し、非対話かつ公開検証可能な証明を効率的に構築する新しいアプローチとして提案された。

本研究の目的は、このVOLEitHプロトコルがブロックチェーン環境において実際にどの程度の性能を発揮するかを評価することである。具体的には、証明生成時間、証明検証時間、証明サイズ、およびそれらのオンチェーン検証におけるガス消費量などの主要な性能指標をベンチマークし、その実用性や効率性を検証することを目的とする。

\subsection{Vector Oblivious Linear Evaluation (VOLE)}
Vector Oblivious Linear Evaluation(VOLE)は、送信者が複数の秘密データの中から受信者が選択した一つのみを伝え、かつ送信者には何が選択されたかを知られないという性質を持つOblivious Transfer (OT) の算術形式と見なされる暗号学的プリミティブであり、証明者と検証者の間で、秘密裡に線形代数的な相関関係を確立する二者間プロトコルである。

$k$をセキュリティパラメータ(暗号システムの安全性の強度を示す数値)、$\mathbb{F}_{2^k}$を要素数$2^k$の有限体とする。プロトコル完了後、証明者は秘密のベクトル $\mathbf{u} \in \mathbb{F}_2^\ell$ とランダムなVOLEタグ $\mathbf{v} \in \mathbb{F}_{2^k}^\ell$ を保持し、検証者はグローバルキー $\Delta \in \mathbb{F}_{2^k}$ とVOLEキー $\mathbf{q} \in \mathbb{F}_{2^k}^\ell$ を保持する。これらの値は、以下のVOLE相関式を満たす。
\begin{equation}
    q_i = u_i \cdot \Delta + v_i \quad (i=0, \dots, \ell-1)
\end{equation}
この相関は、証明者が持つランダムビット $u_i$ に対する\textbf{線形準同型コミットメントスキーム}として機能する。
一般にコミットメントスキームとは、送信者が秘密の値を事前に確定させ(コミット)、後からその値を公開し、そのコミットが偽りでないことを証明できる暗号プリミティブである。この線形準同型コミットメントは以下の性質を持つ。
\begin{description}
    \item[秘匿性 (Hiding)] 検証者は、$q_i$と$\Delta$から$u_i$の情報を得られない。これは、証明者が知るランダムな値$v_i$が$u_i$の値をマスクするためである。
    \item[拘束性 (Binding)] 証明者は、検証者の持つ秘密$\Delta$を知らない限り、一度コミットした$u_i$を不正な値に開示することができない。その成功確率は計算論的に無視できるほど小さい。
\end{description}
VOLEの最も強力な特性は、その線形準同型性にある。同じグローバルキー$\Delta$を持つ複数のVOLE相関 $(u_i, v_i, q_i)$ と公開定数 $c, c_1, \dots, c_n \in \mathbb{F}_{2^k}$ が与えられたとき、証明者と検証者は新しい値 $u' = c + \sum_{i=1}^{n} c_i u_i$ に対するVOLE相関 $(u', v', q')$ を、追加の通信なしにローカルで計算できる。
\begin{description}
    \item[証明者の計算] 証明者は新しいVOLEタグ$v'$を以下のように計算する。
    \begin{equation}
        v' = \sum_{i=1}^{n} c_i v_i
    \end{equation}
    \item[検証者の計算] 検証者は新しいVOLEキー$q'$を以下のように計算する。
    \begin{equation}
        q' = c \cdot \Delta + \sum_{i=1}^{n} c_i q_i
    \end{equation}
\end{description}
算術回路とは、加算や乗算などの算術演算を表すゲートで構成される計算モデルであり、暗号学的証明システムにおいてプログラムの検証対象として用いられる。VOLEが有するこの強力な線形準同型性は、算術回路の検証コストに直接的な影響を及ぼす。
VOLE相関 ($q_i = u_i \cdot \Delta + v_i$) は、その構造自体が線形な関係式である。そのため、加算やスカラー乗法といった\textbf{線形演算}は、VOLEと高い親和性を持つ。例えば、2つの値に対するVOLE相関が与えられたとき、それらの和に対する新しいVOLE相関は、証明者と検証者がそれぞれ手元で対応する値を足し合わせるだけで構築可能である。これは、線形演算がVOLEの線形構造を保存するためであり、追加の通信を一切要しない。

一方で、乗算のような\textbf{非線形演算}は、この線形構造を破壊する。2つのVOLE相関を乗算しても、その結果が有効なVOLE相関となることはない。そのため、証明者は、自身が正しく乗算を実行したことを検証者に納得させるための、ないしは検証させるための追加のプロトコル、すなわち次節で詳述する\textbf{乗算検定}を実行する必要がある。この検定は、非線形な乗算関係の検証を、追加の通信を伴う線形なチェックへと帰着させるものである。したがって、VOLEベースのゼロ知識証明における主要な計算・通信コストは、この非線形演算を処理するための乗算検定に集約される。

\subsection{VOLE-based ZKと乗算検定}
ゼロ知識証明とは、ある命題が真であることを、その命題が真であること以外のいかなる情報も開示することなく証明者が検証者に納得させる暗号技術である。VOLEが有する線形性(加法準同型性)を利用することで、効率的なゼロ知識証明プロトコル(VOLE-based ZK)を構築可能である。このパラダイムは、QuickSilver\cite{quicksilver}などのプロトコルで採用されている。

乗算ゲート、例えば $w_\gamma = w_\alpha \cdot w_\beta$ の検証は、特別な\textbf{乗算検定プロトコル}を必要とする。このプロトコルは、証明者と検証者が保持するVOLE相関が、実際の乗算関係を正しく反映しているかを、検証者の秘密キー$\Delta$を利用して確認するものである。
\begin{description}
    \item[証明者の役割] 証明者は、乗算ゲートに関わる自身の秘密値($u_\alpha, v_\alpha, u_\beta, v_\beta$など)に基づいて、中間値である$a_0$と$a_1$を計算し、検証者に送信する。
    \item[検証者の役割] 検証者は、自身が持つVOLEキー($q_\alpha, q_\beta, q_\gamma$)から特定の値$b$を計算する。その後、証明者から受け取った$a_0, a_1$と、自身の秘密キー$\Delta$を用いて、以下の線形な等式が成立するかを検証する。
    \begin{equation}
        b \stackrel{?}{=} a_0 + a_1 \cdot \Delta
    \end{equation}
    このチェックが成功すれば、証明者が正直に計算を行ったことが高い確率で保証される。もし証明者が不正を働いていた場合、この等式を成立させるには$\Delta$の値を推測する必要があり、その成功確率は計算論的に無視できるほど小さい。
\end{description}

しかし、この仕組みは、検証者がグローバルキー$\Delta$の値を秘密に保つことを前提としている。このため、証明を検証できるのが特定の検証者に限定される\textbf{指定検証者証明(Designated Verifier Proof)}となり、第三者が検証できる\textbf{公開検証可能性(Public Verifiability)}を持たないという課題を有していた。

\subsection{VOLE-in-the-Head}
前節で述べた指定検証者問題を解決し、公開検証可能性を実現するものがVOLE-in-the-Head (VOLEitH) と呼ばれるコンパイラである。
VOLEitHは、SoftSpokenOT \cite{softspokenot} のような効率的なVOLE相関生成プロトコルを、MPC-in-the-Head \cite{mpc-in-the-head} の手法と組み合わせることで、非対話(Non-interactive、すなわち証明生成と検証の間に追加の通信が不要であること)かつ公開検証可能な証明を構築する。

VOLEitHの効率性の根幹を支えるのは、\textbf{SoftSpokenOT}\cite{softspokenot}と呼ばれるプロトコルである。これは、少数の高価なベースOT(Oblivious Transfer)から、擬似乱数を用いて極めて多数の安価なOTインスタンスを生成するOT拡張技術の一種である。
特に、相関のある乱数を利用してOTをバッチ処理することに特化しており、VOLE相関を生成する上で必要となる「$N$個のシードの中から、チャレンジで指定された1個のシードを除いた$N-1$個の情報を効率的に転送する」という処理(All-but-One OT)と高い親和性を持つ。
結果として、SoftSpokenOTはVOLEitHにおけるVOLE相関の生成エンジンとして機能し、プロトコル全体の通信・計算コストを大幅に削減する役割を担う。

その核心は、検証者が秘密のグローバルキー$\Delta$を持つ代わりに、証明者自身がプロトコルの中で$\Delta$の元となる情報にコミットし、後からFiat-Shamir変換によってチャレンジとして$\Delta$を導出する点である。Fiat-Shamir変換とは、対話型の証明プロトコルを、チャレンジがハッシュ関数によって生成される非対話型の証明に変換する一般的な手法である。
このプロセスは、GGMツMリー\cite{ggm}を用いたAll-but-One Oblivious Transfer (OT) の技術を用いて効率的に実装される。

\paragraph{GGMツリーによるAll-but-Oneコミットメント}

GGMツリー(Goldreich-Goldwasser-Micali Tree)は、短い入力(シード)から長い擬似乱数系列を効率的に生成するための仕組みであり、特に擬似乱数生成器(PRG)を繰り返し適用することで構築される木構造である。
このツリーの各ノードは、親ノードのシードからPRGを用いて計算され、再帰的に展開される。
最終的に最下層に到達した「葉」(リーフノード)には、一連の擬似乱数が対応付けられる。
GGMツリーの重要な特性は、この構造を利用して、ある特定の一つのリーフノード以外の全てのリーフノードの値を効率的に開示できる点にある。この性質が、All-but-Oneコミットメントスキームを構築する上で不可欠となる。

All-but-Oneコミットメントは、コミッターが複数の値の集合(例えば、$N$個の値)に対してコミットメントを行い、後からその集合内の任意の一つの値**以外**の全ての値を公開できる暗号プリミティブである。このとき、公開されなかった一つの値については、コミットメントの受信者には全く情報が漏洩しないことが保証される。

VOLE-in-the-Head (VOLEitH) プロトコルでは、このGGMツリーを用いたAll-but-Oneコミットメントが、特に$N$個のシードの中から、チャレンジで指定された1個のシードを除いた$N-1$個の情報を効率的に転送する(All-but-One OTの文脈で)ために利用される。
これにより、検証者が秘密のグローバルキー$\Delta$を持つ代わりに、証明者自身が$\Delta$の元となる情報にコミットし、公開検証可能性を実現する。

プロトコルは以下のように進行する。
\begin{description}
    \item[ステップ1: GGMツリー構築とコミットメント]
    まず、セキュリティパラメータ$\lambda$に対して、入力シードを2倍の長さの出力に拡張する擬似乱数生成器 (PRG) $G: \{0,1\}^\lambda \to \{0,1\}^{2\lambda}$ を定義し、$G(s) = (s_0, s_1)$ とする。証明者は、ランダムなルートシード$s_{root} \in \{0,1\}^\lambda$を選択する。
    深さ$d = \log N$のGGMツリーは、このルートシードからPRGを再帰的に適用することで構築される。各ノードは、その親ノード$s_{parent}$から$G(s_{parent})$によって2つの子ノードシード($s_0, s_1$)が生成される形で展開される。具体的には、インデックス$i \in \{0, \ldots, N-1\}$の$d$ビット表現を$i_1i_2\dots i_d$とすると、リーフ(葉)シード$sd_i$は、ルートからパス$i_1\dots i_d$を辿って$sd_i = s_{i_1\dots i_d}$として計算される。
    証明者は、こうして得られた$N$個の全てのリーフシード$\{sd_i\}_{i=0}^{N-1}$に対してハッシュ関数$H$を適用し、ハッシュ値$h_{com}=H(sd_0, \ldots, sd_{N-1})$を計算して、これをコミットメントとして公開する。

    \item[ステップ2: VOLE値の計算]
    証明者は、各リーフシード$sd_i$をさらに別のPRG($G'$とする)で展開してランダムな文字列$r_i = G'(sd_i)$を得て、自身の秘密VOLE値$\mathbf{u}, \mathbf{v}$を以下のように計算する。
    \begin{align}
        \mathbf{u} &= \sum_{i=0}^{N-1} r_i \\
        \mathbf{v} &= \sum_{i=0}^{N-1} i \cdot r_i
    \end{align}

    \item[ステップ3: チャレンジと開示 (All-but-One OT)]
    証明の主要部分が構成された後、Fiat-Shamir変換により、証明全体のトランスクリプトからチャレンジ$\Delta \in \{0, \ldots, N-1\}$が導出される。この$\Delta$が開示を免除される(すなわち秘匿される)インデックスとなる。
    証明者は、GGMツリーにおいて$sd_\Delta$を隠したまま残りの全てのシードを開示するため、ルートから$sd_\Delta$に至るパス上の各ノードの\textbf{兄弟ノード}(sibling node)のシードを開示情報$pdecom$として証明に含める。$\Delta$のビット表現を$\Delta_1\dots\Delta_d$とすると、$pdecom$は以下の$\log N$個のシードの集合である。
    \begin{equation}
        pdecom = \{ s_{\Delta_1\dots\Delta_{j-1}\bar{\Delta}_j} \}_{j=1}^d, \quad \text{ここで } \bar{\Delta}_j = 1-\Delta_j
    \end{equation}
    この$pdecom$に含まれる兄弟ノードのシードは、$\Delta$に続くパスとは異なる全てのパスを再構築するのに必要な情報を提供する。

    \item[ステップ4: 検証]
    検証者は、まず証明者と同様にFiat-Shamir変換で$\Delta$を導出する。次に、$pdecom$に含まれる$\log N$個の兄弟ノードのシードを用いて、$\Delta$以外の全てのインデックス$i \neq \Delta$に対応するリーフシード$sd_i$を再計算する。この再計算が可能であるのは、$i$と$\Delta$のビット表現が最初に異なる位置を$j$とすると、$sd_i$の計算に必要なプレフィックス$i_1\dots i_j$を持つノードのシードが$pdecom$に含まれる$s_{\Delta_1\dots\Delta_{j-1}\bar{\Delta}_j}$(ここで $i_j = \bar{\Delta}_j$)から計算を開始できるためである。
    検証者は、復元した$N-1$個のリーフシードと、公開されている$h_{com}$が正しく再構成できることを確認し、開示の正当性を検証する。最後に、これらのシードと$\Delta$を用いてVOLEキー$\mathbf{q}$を計算し、VOLE相関$q_i = u_i \Delta + v_i$の検証へと進む。
\end{description}

\subsection{Groth16}
まず、Groth16を理解するために、zk-SNARKs(zero-knowledge Succinct Non-interactive ARguments of Knowledge)について説明する。
zk-SNARKsは、以下の性質を持つ暗号学的証明システムである。
\begin{itemize}
    \item \textbf{ゼロ知識 (Zero-Knowledge)}: 証明が「あるステートメントが真である」という事実以外のいかなる情報も開示しない。例えば、秘密の入力値を知っていることを証明しても、その秘密の値自体は明かされない。
    \item \textbf{簡潔性 (Succinctness)}: 証明のサイズが非常に小さく、検証時間が短い。これにより、検証コストが重要なブロックチェーンなどの環境に適している。
    \item \textbf{非対話性 (Non-interactive)}: 事前の一度きりのセットアップの後、証明者と検証者の間でインタラクションなしに証明が生成・検証される。
    \textbf{知識の論拠 (ARgument of Knowledge)}: 証明者が実際に秘密の入力(witness)を知っていることを保証する。
\end{itemize}
Groth16は、このようなzk-SNARKsの中でも特に効率的で広く利用されている方式の一つであり、特にその証明サイズの小ささと検証の速さで知られている。
ブロックチェーンのように検証コストが重視される環境においては、特に強力な選択肢となる。

Groth16プロトコルは主に3つのアルゴリズムから構成される:
Groth16は、算術回路として表現された計算を、R1CS(Rank-1 Constraint System)と呼ばれる制約システムに変換し、さらにこれをQuadratic Arithmetic Program (QAP) へと変換することで証明を構築する。QAPは、ある多項式が特定の根を持つかどうかを検証する問題に帰着され、これにより複雑な計算の正当性を効率的に検証可能となる。

このプロトコルは、楕円曲線上のペアリング(双線形写像)の性質を最大限に活用し、高い効率性を実現している。
\begin{description}
    \item[Setup ($\mathcal{C} \to (\text{pk}, \text{vk})$)]
    このフェーズでは、証明対象の計算をR1CS形式で表現された算術回路 $\mathcal{C}$ から、証明鍵(Proving Key, pk)と検証鍵(Verification Key, vk)を生成する。このプロセスは回路ごとに一度だけ実行され、通常「信頼できるセットアップ(Trusted Setup)」として知られる。このセットアップでは、ランダムに選択された秘密のパラメータ $s, \alpha, \beta, \gamma, \delta \in \mathbb{F}_p^*$ を用いて、共通参照文字列(Common Reference String, CRS)が生成される。このCRSには、$G_1, G_2$ 上の様々な群要素の組 $([s^i]_1, [s^i]_2)$, $([\alpha]_1, [\alpha]_2)$, $([\beta]_1, [\beta]_2)$, $([\gamma]_1, [\gamma]_2)$, $([\delta]_1, [\delta]_2)$, $([\beta s^i]_1)$, $([\gamma s^i]_1)$, $([s^i \text{poly}(x)]_1)$ などが含まれる。ここで $\mathbb{G}_1, \mathbb{G}_2$ は楕円曲線上の点からなる群であり、$[X]_1, [X]_2$ はそれぞれ $\mathbb{G}_1, \mathbb{G}_2$ における $X$ の楕円曲線乗算を意味する。このCRS生成時に用いられた秘密のパラメータ(「有毒廃棄物 (toxic waste)」と呼ばれる)が漏洩すると、不正な証明を生成可能になるため、その破棄が極めて重要である。
    \item[Prove ($\text{pk}, x, w \to \pi$)]
    証明者は、証明鍵 pk、公開入力 $x$、および秘密入力 $w$ を用いて、命題が真であることの証明 $\pi = (A, B, C)$ を生成する。この証明は、QAPの解に対応する特定の多項式 $h(s)$ やその他のランダム性 $r, t \in \mathbb{F}_p^*$ を用いて、$\mathbb{G}_1$ および $\mathbb{G}_2$ 上の3つの群要素 $A \in \mathbb{G}_1, B \in \mathbb{G}_2, C \in \mathbb{G}_1$ として構築される。具体的には、$A = [A_w(s) + r\delta]_1$, $B = [B_w(s) + t\delta]_2$ または $B = [B_w(s) + t\delta]_1$, $C = [C_w(s) + (h(s)T(s) + \frac{A_w(s) \beta + B_w(s)\alpha + \gamma}{r\delta})]_1$ のような形式をとる(具体的な形式はGroth16論文参照)。ここで $A_w(s), B_w(s), C_w(s)$ は公開入力と秘密入力 $w$ から導出される多項式である。
    \item[Verify ($\text{vk}, x, \pi \to \{\text{True}, \text{False}\}$)]
    検証者は、検証鍵 vk、公開入力 $x$、および証明 $\pi = (A, B, C)$ を受け取り、その正当性を検証する。検証は、CRSに含まれる要素と証明 $\pi$ の群要素を用いて、以下のペアリング方程式が成立するかどうかを確認することによって行われる。
    \begin{equation}
        e(A, B) = e([\text{A_input}(x)]_1, [\text{B_input}(x)]_2) \cdot e(C, [\delta]_2)
    \end{equation}
    ここで $e$ は楕円曲線上の双線形ペアリング演算を意味する。この検証は、回路の規模に関わらず固定長の証明と数回のペアリング演算で行われるため、極めて高速である。
\end{description}

Groth16の証明は常に一定のサイズであり、通常3つの群要素($\mathbb{G}_1$の元2つ、$\mathbb{G}_2$の元1つ)で構成される。この効率性の代償として、回路ごとに異なるTrusted Setupが必要であり、汎用性(universality)に欠けるというトレードオフが存在する。
