\section{前提知識と関連研究}
本章では、後続のベンチマークと考察を理解するために必要な暗号学的前提を、数式レベルで簡潔に整理する。
特に、VOLEによるMAC相関、MPC-in-the-Headの健全性構造、GGM木に基づくベクトルコミットメント、そしてVOLE-in-the-Head (VOLEitH) への写像を強調する。

\subsection{Vector Oblivious Linear Evaluation (VOLE)}
有限体 $F$ 上で、送信者はベクトル $\mathbf{u}, \mathbf{v} \in F^m$ を、受信者はグローバル鍵 $\mathbf{\Delta} \in F^m$ を入力とする。
プロトコル終了後、受信者は
\begin{equation}
    \mathbf{q} = \mathbf{u} \circ \mathbf{\Delta} - \mathbf{v} \in F^m
\end{equation}
を満たすペア $(\mathbf{u}, \mathbf{q})$ を得る($\circ$ は要素積)。
$\mathbf{q}$ は $\mathbf{u}$ に対する情報理論的MACとなり、$\mathbf{\Delta}$ を知らずに $\mathbf{q}$ を固定したまま $\mathbf{u}$ を改ざんすることはできない(binding)。
この性質が後述の健全性担保に直接利用される。

\paragraph{サブフィールドVOLE}
FAESTなどでは、入力 $\mathbf{u}$ を $\mathbb{F}_2$ に制限しつつ、$\mathbf{v}, \mathbf{q}, \mathbf{\Delta}$ を拡大体 $\mathbb{F}_{2^\lambda}$ 上に置くサブフィールドVOLEが用いられる。
これはビット演算を保ちつつ、MACは大きな体で保持することで効率と安全性を両立するためである。

\subsection{MPC-in-the-Head (MPCitH)}
NPリレーション $R(x, w)$ に対し、証明者 $P$ は証人 $w$ を $N$ 份のシェア $w_i$ に分割し(加法秘密分散等)、$f(x, w_1,\dots,w_N)=R(x,\bigoplus_i w_i)$ を計算する半正直MPC $\Pi_f$ を頭の中でエミュレートする。
識別プロトコルは次の4段階で構成される。
\begin{enumerate}
    \item \textbf{ビュー生成とコミット}: $\Pi_f$ の各仮想パーティのビュー $V_1,\dots,V_N$ にコミットする。
    \item \textbf{チャレンジ}: 検証者 $V$ はインデックス $i^* \in \{1,\dots,N\}$ をランダムに選ぶ。
    \item \textbf{開示}: $P$ は $i^*$ を除く $N-1$ 個のビューを開示する。
    \item \textbf{検証}: $V$ は開示ビューの整合性と、$\Pi_f$ の実行規則への準拠を検査する。
\end{enumerate}
$\Pi_f$ が $(N-1)$-privacy を満たすとき、開示されなかった $V_{i^*}$ を偽造しない限り不正な $P$ は受理されないため、健全性が得られる。
通信量は回路サイズ $s$ に線形($O(s) + \mathrm{poly}(k,\log s)$)で、汎用性が高い。

\subsection{VOLE-in-the-HeadとGGMベクトルコミットメント}
VOLEitHは、MPCitHの「開示しない1パーティ」を、VOLEのグローバル鍵 $\mathbf{\Delta}$ に対応付けることで、指定検証者型のVOLE系プロトコルを公開検証可能にする。
鍵となるのが、長さ倍増PRGから構成するGGM木に基づく All-but-One ベクトルコミットメント (VC) である。
\begin{itemize}
    \item 高さ $h=\log N$ のGGM木の根シードから $N$ 個の葉シード $sd_i$ を生成し、各葉にVOLE用の乱数列 $r_i$ を割り当てる。
    \item 開示しないインデックス $j^*$ を除き、$P$ は各葉への経路で必要な兄弟ノードのシードを開示するだけで $N-1$ 個の $r_i$ を再現させられる(通信量は $O(\log N)$)。
    \item 開示されない $r_{j^*}$ が $\mathbf{\Delta}$ に対応し、$V$ は公開された $r_i$ の線形結合から $\mathbf{q}=\mathbf{u}\circ\mathbf{\Delta}-\mathbf{v}$ を再構成する。
\end{itemize}
この構成により、OTを用いずに非対話でVOLE相関を生成し、MACのbinding性を保ったままFiat--Shamir変換を適用できる。

\subsection{QuickSilverと乗算検証}
QuickSilverはLine-Point Zero-Knowledgeパラダイムを用い、VOLE MACの加法準同型性で乗算制約 $w_\alpha \cdot w_\beta = w_\gamma$ を検査する。
証明者は応答を $\tilde{a}_0,\tilde{a}_1,\tilde{b}$ にマスクし、検証者は VOLE 関係
\begin{equation}
    \tilde{b} = \tilde{a}_0 + \tilde{a}_1 \cdot \Delta
\end{equation}
が成立することを確認するだけで健全性を得る。
VOLEitHは、このQuickSilver型の検査をGGMベースのVCと組み合わせ、公開検証を実現している。

\subsection{SNARKとオンチェーン検証の概観}
Groth16を例に、オンチェーン検証は (i) 証明と公開入力をcalldataで受領、(ii) 回路固有の検証鍵をコントラクトに格納、(iii) 公開入力とICベクトルのMSMで\(\mathrm{vk\_x}\)を計算、(iv) ペアリングprecompileで等式を判定、の4段階からなる。
ガスは主にペアリング(1回あたり\(\sim\)34k gas、合計15--20万gas規模)と公開入力数に比例するMSMで支配され、calldata課金は16 gas/非ゼロバイトで証明サイズ1055バイトなら約1.7万gasとなる。
コードサイズは24KB上限のため検証鍵の配置も設計上の制約となる。

\subsection{設計指針(本稿で用いる前提)}
\begin{itemize}
    \item \textbf{証明サイズの固定性}: Groth16の証明サイズは一定(本稿では1055バイト)で、回路規模が増えてもcalldata課金は増えない。
    \item \textbf{公開入力の削減}: MSMコストは公開入力数に線形。ハッシュ圧縮等で項目数を抑える。
    \item \textbf{検証鍵の配置}: コントラクトサイズ上限を考慮し、検証鍵を分離・共有する設計をとる。
    \item \textbf{フィールド整合性}: 証明生成・検証ともBN254を用いる。別曲線を選ぶ場合は専用プリコンパイルやロールアップを前提とする。
\end{itemize}
