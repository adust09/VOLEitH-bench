\section{結論と今後の展望 (Conclusion and Future Work)}

\subsection{結論}
本研究では、証明者効率の高いVOLE-in-the-Head(VOLEitH)プロトコルと、証明圧縮に優れたSNARKを組み合わせたハイブリッドアーキテクチャを構築し、そのオンチェーン検証における実現可能性と性能を定量的に評価した。

ベンチマークを通じて、以下の主要な知見が得られた。
\begin{enumerate}
    \item \textbf{VOLEitHの基本的なトレードオフ}: VOLEitHは、CircomのようなR1CSベースのシステムと比較して、証明者の計算を最大15.5倍高速化する一方で、生成される証明サイズが6000倍以上も増大することが確認された。この巨大な証明は、単体ではオンチェーン検証の大きな障壁となる。
    \item \textbf{SNARK圧縮の有効性}: VOLEitHの証明をSNARK(Groth16)で圧縮することにより、回路の複雑さに関わらず、最終的な証明サイズを1,055バイト、オンチェーン検証コストを約21万ガスに固定化できることを示した。これにより、VOLEitHのオンチェーン応用の道が拓かれる。
    \item \textbf{アーキテクチャのボトルネック}: エンドツーエンドのプロセスにおける主要なボトルネックは、SNARKの証明生成時間であり、これは回路内の乗算(ANDゲート)の数に起因する制約数に強く依存することが明らかになった。
\end{enumerate}
結論として、VOLEitHとSNARKを組み合わせたハイブリッドアプローチは、「クライアントサイドでの高速な証明生成」と「低コストなオンチェーン検証」という、一見すると相反する要求を両立しうる有望なアーキテクチャである。本研究は、その具体的な性能データとトレードオフを明らかにすることで、このアプローチの実用性に関する最初のマイルストーンを提示した。

\subsection{今後の展望}
本研究の成果を踏まえ、今後の展望として以下の方向性が考えられる。
\begin{itemize}
    \item \textbf{性能改善}:
    \begin{itemize}
        
        \item \textbf{SNARK証明生成の高速化}: 本アーキテクチャの主要なボトルネックであるSNARK証明生成時間を短縮するため、VOLEitHからR1CSへのより効率的な変換手法の研究や、Plonk/Halo2のような新しいSNARKシステムとの組み合わせを検討する必要がある。特に、Universal SNARKsを用いることで、開発の柔軟性向上も期待できる。
        
        \item \textbf{ガス代のさらなる削減}: 本研究では約21万ガスであった検証コストを、Verifierスマートコントラクトのさらなる最適化や、将来的なEVMのプリコンパイル追加などによって削減する研究が期待される。
    \end{itemize}
    \item \textbf{応用展開}:
    \begin{itemize}
        
        \item 本研究では基本的な回路を用いたが、今後は分散型ID(dID)、プライベートトランザクション、オンチェーンゲームのプライベート状態更新といった、より複雑で実用的なアプリケーションへの適用評価を進めるべきである。これにより、実際のユースケースにおける本アーキテクチャの有効性と課題がさらに明確になる。
    \end{itemize}
    \item \textbf{セキュリティの深化}:
    \begin{itemize}
        
        \item 本研究で採用したVOLEitHはLPN仮定に基づくポスト量子耐性を持つが、SNARK(Groth16)は量子コンピュータに対して脆弱である。エンドツーエンドでのポスト量子耐性を実現するため、STARKsのような耐量子性を持つ証明システムとの組み合わせを検討することは重要な研究課題である。また、アーキテクチャ全体としてのセキュリティレベルの形式的な分析も今後の課題となる。
    \end{itemize}
\end{itemize}
