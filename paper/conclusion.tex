\section{結論と今後の展望 (Conclusion and Future Work)}
本論文では主にGroth16を採用したが、今後の展望として、FAEST/FAESTER(高速なポスト量子署名スキーム)、Binius(二進演算に特化したZKPシステム)、およびUniversal SNARK(単一のセットアップで任意の回路に対応可能なSNARKs、例:Plonk/Halo2)などの代替技術の検討が挙げられる。

\subsection{結論}
本研究では、証明者効率の高いVOLE-in-the-Head(VOLEitH)プロトコルと、証明圧縮に優れたSNARKを組み合わせたハイブリッドアーキテクチャを構築し、そのオンチェーン検証における実現可能性と性能を定量的に評価した。

ベンチマークを通じて、以下の主要な知見が得られた。
\begin{enumerate}
    \item \textbf{VOLEitHの基本的なトレードオフ}: VOLEitHは、証明者の計算をマイクロ秒からミリ秒オーダーで実行可能であり、その高速性が確認された。一方で、生成される証明サイズは、基本的な回路であっても数十KBから数百KBに達し、オンチェーン検証に用いられるSNARK証明(1KB)と比較して依然として大きく、その圧縮の必要性が再確認された。
    この巨大な証明は、単体ではオンチェーン検証の大きな障壁となる。
    \item \textbf{SNARK圧縮の有効性}: VOLEitHの証明をSNARK(Groth16)で圧縮することにより、回路の複雑さに関わらず、最終的な証明サイズを1,055バイト、オンチェーン検証コストを約21万ガスに固定化できることを示した。
    これにより、VOLEitHのオンチェーン応用の道が拓かれる。
    \item \textbf{アーキテクチャのボトルネック}: エンドツーエンドのプロセスにおける主要なボトルネックは、SNARKの証明生成時間であり、これは回路内の乗算(ANDゲート)の数に起因する制約数に強く依存することが明らかになった。
\end{enumerate}
結論として、VOLEitHとSNARKを組み合わせたハイブリッドアプローチは、「クライアントサイドでの高速な証明生成」と「低コストなオンチェーン検証」という、一見すると相反する要求を両立しうる有望なアーキテクチャである。
本研究は、その具体的な性能データとトレードオフを明らかにすることで、このアプローチの実用性に関する最初のマイルストーンを提示した。

\subsection{今後の展望}
本研究の成果を踏まえ、特に優先度の高い研究課題は以下の3点である。
\begin{enumerate}
    \item \textbf{Field Mapping最適化}: Mystique型のデータ変換やLookup TableをVOLEitHに取り入れ、$\mathbb{F}_2$と$\mathbb{F}_p$間の射像コストを削減する。
    \item \textbf{GGM木再構成の高度化}: FAESTERのような最適化とFoldingスキームを組み合わせ、検証フェーズに必要な制約数を抑制する。
    \item \textbf{代替SNARK/証明システムの検討}: BiniusやRecursive SNARKなど、二進演算に適した新しい証明システムを評価し、ボトルネックの抜本的解決を目指す。
\end{enumerate}

これらに加えて、以下のエンジニアリング課題にも継続的に取り組む必要がある。
\begin{itemize}
    \item \textbf{SNARK証明生成の最適化}: VOLEitHからR1CSへの変換フローを高速化し、Plonk/Halo2といったUniversal SNARKやSolidity verifierのガス最適化を検討する。
    \item \textbf{アプリケーション駆動の評価}: 分散型ID、プライベートトランザクション、オンチェーンゲームなど実アプリケーションでエンドツーエンド評価を行い、制約プロファイルとUX要件を明確化する。
\end{itemize}
