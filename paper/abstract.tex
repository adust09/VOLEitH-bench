\begin{abstract}
Vector Oblivious Linear Evaluation (VOLE) は、二者間で線形関係を秘匿したまま、ある種の相関を生成するための暗号学的プリミティブであり、これを応用したゼロ知識証明 VOLE-in-the-Head (VOLE-itH) は、従来のゼロ知識証明と比較して線形演算をVOLEプリプロセスに置き換えることで証明生成コストを大幅に削減する。
一方、Ethereum での公開検証ではオンチェーン検証コストが実用性のボトルネックとなる。本研究は、VOLE-itH のオンチェーン適用に向けた実装ベースの評価として、(1) 証明生成・検証の計算量と証明サイズを測定し、(2) SNARK (Groth16) で圧縮した上で Ethereum verifier を実装し、そのガスコストを評価した。
SHA-256/Keccak-F/基本論理回路でベンチマークした結果、VOLE-itH の証明生成は Circom 実装より最大 15.5× 高速だが証明サイズは 6000× 増大した。SNARK で圧縮すると証明サイズを 1,055 バイトに固定でき、オンチェーン検証は実測で 208,967 gas で完了した。
これらの結果から、VOLE-itH をブロックチェーン応用に適用する際のコスト・圧縮トレードオフと実用条件を明らかにする。
\end{abstract}
