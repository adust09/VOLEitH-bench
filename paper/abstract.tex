\begin{abstract}
VOLE-in-the-Head(VOLEitH)は、線形演算を中心とした構成により証明者計算を大幅に削減し、ゼロ知識証明をクライアントデバイスでも実行しやすい点から注目を集めている。
一方、ブロックチェーン業界では、秘匿送金などでゼロ知識証明技術が採用されており、VOLE itHのような軽量な証明システムが求められている。
本研究では、こうした理由からVOLEitHゼロ知識証明のパブリックブロックチェーン上での実用性を評価し、証明生成から検証までのエンドツーエンド(E2E)プロセスにおける計算コスト、証明サイズ、そしてEthereum上でのガス代を詳細に測定・分析した。
ベンチマークとして、標準的な暗号学的ハッシュ関数(SHA-256, Keccak-F)および基本的な論理ゲート回路を用いた。
結果として、VOLEitHは既存のゼロ知識証明実装(Circom)と比較して証明生成が最大15.5倍高速である一方、証明サイズが6000倍以上増大することが確認された。
さらに、VOLEitHの証明をSNARKで圧縮することにより、証明サイズを1,055バイトに固定し、オンチェーン検証のガス代を約21万ガスに抑えられることを示した。
本稿は、VOLEitHをオンチェーンアプリケーションへ適用する際の技術的なトレードオフを明らかにし、将来的なスケーラビリティ向上のための示唆を与える。
\end{abstract}
