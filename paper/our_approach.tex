\section{ベンチマーク設計}
本章では、VOLEitHとSNARKを組み合わせたアーキテクチャの性能を定量的に評価するために設計したベンチマークの詳細について述べる。

\subsection{提案アーキテクチャ:VOLEitH証明のSNARKによる圧縮}
\label{subsec:our_approach}
本研究で評価するアーキテクチャは、VOLE-in-the-Head(VOLEitH)が生成する巨大な証明サイズの問題を解決し、オンチェーン検証に適した形式に圧縮することを目的としている。
このため、証明者効率に優れたVOLEitHと、証明の簡潔性および検証効率に優れたzk-SNARK(本研究ではGroth16\cite{groth16}を採用)を組み合わせたハイブリッドアプローチを採用する。
このアーキテクチャの究極的な目的は、VOLEitHの利点である「クライアントサイドでの高速な証明生成」と、SNARKの利点である「低コストなオンチェーン検証」を両立させることにある。

証明者が最終的なオンチェーン検証用の証明を生成するまでのプロセスは、以下の4つのステップで構成される。

\begin{enumerate}
    \item \textbf{ステップ1:VOLEitH証明の生成}
    まず、証明者は与えられた計算(算術回路)に対し、VOLEitHプロトコルを用いて証明\(\pi_{\text{VOLEitH}}\)を生成する。
    このプロセスは、対称鍵暗号ベースの軽い計算で構成されるため非常に高速だが、生成される証明\(\pi_{\text{VOLEitH}}\)のサイズは数MBオーダーと非常に大きい。

    \item \textbf{ステップ2:VOLEitH検証回路のR1CS化}
    次に、ステップ1で生成された\(\pi_{\text{VOLEitH}}\)の正当性を検証するアルゴリズム自体を、一つの計算と見なす。
    この検証ロジックを、SNARKが扱うことができる形式であるR1CS(Rank-1 Constraint System)の算術回路として表現する。
    この「検証回路」は、\(\pi_{\text{VOLEitH}}\)を入力として受け取り、それが有効であれば真を、無効であれば偽を出力する。

    \item \textbf{ステップ3:SNARK証明の生成}
    証明者は、ステップ2で構築した「VOLEitH検証回路」に対して、Groth16プロトコルを用いて証明\(\pi_{\text{SNARK}}\)を生成する。
    この\(\pi_{\text{SNARK}}\)は、「ある有効な\(\pi_{\text{VOLEitH}}\)を知っており、それに対する検証計算が正しく実行された」という事実を証明するものである。
    Groth16の特性により、\(\pi_{\text{SNARK}}\)は回路の複雑さに関わらず一定の小さなサイズ(約1KB)となる。

    \item \textbf{ステップ4:オンチェーン検証}
    最終的に、コンパクトなSNARK証明\(\pi_{\text{SNARK}}\)と、計算の公開入力のみがEthereum上の検証者スマートコントラクトに送信される。
    スマートコントラクトは、事前にデプロイされた検証鍵を用いて\(\pi_{\text{SNARK}}\)を検証する。この検証は、数回のペアリング演算で完了するため、極めて少ないガス代で実行可能である。
\end{enumerate}

この一連のプロセスにより、クライアント側では証明生成の大部分を軽量なVOLEitHプロトコルが担い、ブロックチェーン側では検証コストの高い計算がコンパクトなSNARK証明の検証に置き換えられる。
本研究では、このアーキテクチャをRust言語で実装した。VOLEitHの実装には\texttt{schmivitz}ライブラリを、SNARK回路の構築と証明生成には\texttt{arkworks}エコシステムを利用し、オンチェーン検証用のSolidityコントラクトは\texttt{ark-groth16}の機能を用いて生成した。

\subsection{測定項目}
本研究では、証明システムの性能と実用性を多角的に評価するため、以下のメトリクスを測定対象とした。
\begin{table}[hbtp]
    \centering
    \begin{tabular}{|l|l|l|}
        \hline
        \textbf{メトリクス} & \textbf{説明} & \textbf{単位/備考} \\
        \hline
        証明生成時間 & 証明者が、ある計算に対する証明を生成するために要する時間 & ms または $\mu$s \\
        \hline
        証明検証時間 & 検証者が、与えられた証明の正当性を検証するために要する時間 & ms または $\mu$s \\
        \hline
        証明サイズ & 生成された証明データの大きさ & バイト(B) \\
        \hline
        通信オーバーヘッド & 非対話型証明において、証明者が検証者に送信する必要がある総データ量。基本的に証明サイズとほぼ同等 & バイト(B) \\
        \hline
        計算負荷 & 証明生成および検証プロセス中に消費されるCPU使用率および最大メモリ使用量 & CPU使用率(\%)、メモリ(MB) \\
        \hline
        SNARK制約数 & VOLEitHの証明をSNARKに変換する際に生成されるR1CSの制約数。証明生成時間に大きく影響 & - \\
        \hline
        オンチェーン検証ガス代 & 生成されたSNARK証明をEthereumのスマートコントラクトで検証する際に消費されるガス量 & gas \\
        \hline
    \end{tabular}
    \caption{ベンチマーク測定項目一覧}
\end{table}

\subsection{評価環境}
すべてのベンチマークは、以下の統一された環境で実施した。
\begin{itemize}
    \item \textbf{ハードウェア}:
    \begin{itemize}
        \item CPU: Apple M1
        \item メモリ: 16GB
    \end{itemize}
    \item \textbf{ソフトウェア}:
    \begin{itemize}
        \item 言語: Rust
        \item VOLEitH実装: \texttt{schmivitz} ライブラリ
    \end{itemize}
\end{itemize}

\subsection{評価対象回路}
本ベンチマークでは、プロトコルの基本的な性能と、より実践的な応用における性能の両方を評価するため、2種類の回路セットを用いた。
\begin{itemize}
    \item \textbf{SHA256回路}:
    \begin{itemize}
        \item 内容: \textbf{SHA-256} 。これらは暗号技術で広く利用される標準的なハッシュ関数であり、複雑な計算の代表例として用いた。なお、本研究では標準的なハッシュ関数を評価対象としたが、近年ではPoseidonのようにSNARK証明系に特化して算術化を効率化したZKフレンドリーなハッシュ関数も提案されている\cite{poseidon}。
        \item 形式: これらの回路は、\href{https://github.com/GaloisInc/swanky/tree/dev/bristol-fashion/circuits}{Bristol Fashion}形式で記述されたものを、本研究で利用するVOLEitHライブラリに適した形式に変換して使用した。
        \item 目的: VOLEitHプロトコル単体の性能と、既存のZKP実装(Circom)との比較評価に用いる。
    \end{itemize}
    \item \textbf{E2E評価用基本回路}:
    \begin{itemize}
        \item 内容: \textbf{100ゲート}および\textbf{1000ゲート}の\textbf{ADD(加算)回路}と\textbf{AND(乗算)回路}。
        \item 目的: エンドツーエンド(E2E)の性能評価、特に回路の規模(ゲート数)と種類(加算/乗算)が、VOLEitHフェーズとSNARKフェーズの各メトリクスにどのような影響を与えるかを詳細に分析するために用いる。
    \end{itemize}
\end{itemize}
