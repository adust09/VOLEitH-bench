\section{ベンチマーク設計}
本章では、VOLEitHとSNARKを組み合わせたアーキテクチャの性能を定量的に評価するために設計したベンチマークの詳細について述べる。

\subsection{提案アーキテクチャ:VOLEitH証明のSNARKによる圧縮}
\label{subsec:our_approach}
ブロックチェーン上でのプライバシー保護に不可欠なゼロ知識証明技術において、VOLE-in-the-Head(VOLEitH)は極めて高速な証明生成を可能にする一方で、その証明サイズが数MBから数十MBに及ぶため、Ethereumのようなリソース制約の厳しいオンチェーン環境での直接検証には適さないという課題を抱えている。一方、zk-SNARKsは証明の簡潔性と検証効率に優れるが、証明生成に高い計算コストを要する。本研究で評価するアーキテクチャは、この両者の課題を解決し、VOLEitHが生成する巨大な証明サイズの問題を解決しつつ、オンチェーン検証に適したコンパクトな形式に圧縮することを目的としている。
このため、証明者効率に優れたVOLEitHと、証明の簡潔性および検証効率に優れたzk-SNARK(本研究ではGroth16\cite{groth16}を採用)を組み合わせたハイブリッドアプローチを採用する。本アプローチの究極的な目的は、VOLEitHの利点である「クライアントサイドでの高速な証明生成」と、SNARKの利点である「低コストかつ予測可能なガス代でのオンチェーン検証」を両立させ、Web3アプリケーションにおけるプライバシー保護の課題を解決するための実用的なソリューションを提供することにある。

% \todo{提案アーキテクチャのフローを示すブロックダイアグラムを挿入}
% 図のキャプション例: 図1: VOLEitH証明のSNARKによる圧縮アーキテクチャの概要
% 図の説明例: 本図は、VOLEitHとSNARKを組み合わせたハイブリッド証明生成・検証プロセスの全体像を示している。クライアント側でVOLEitH証明が生成された後、その検証ロジックがSNARK回路として表現され、コンパクトなSNARK証明が生成される。最終的に、SNARK証明のみがオンチェーンで検証されることで、効率的なプライバシー保護型計算の検証を実現する。

証明者が最終的なオンチェーン検証用の証明を生成するまでのプロセスは、以下の4つのステップで構成される。

\begin{enumerate}
    \item \textbf{ステップ1:VOLEitH証明の生成}
    まず、証明者は与えられた計算(算術回路)に対し、VOLEitHプロトコルを用いて証明\(\pi_{\text{VOLEitH}}\)を生成する。
    このプロセスは、対称鍵暗号ベースの軽い計算で構成されるため非常に高速だが、生成される証明\(\pi_{\text{VOLEitH}}\)のサイズは数MBオーダーと非常に大きい。

    \item \textbf{ステップ2:VOLEitH検証回路のR1CS化}
    次に、ステップ1で生成された\(\pi_{\text{VOLEitH}}\)の正当性を検証するアルゴリズム自体を、一つの計算と見なす(いわゆる「証明の証明」の概念)。これは、既存の巨大なVOLEitH証明を直接オンチェーンで検証する代わりに、その「検証ロジックが正しく実行されたこと」をSNARKで証明するための重要な中間ステップである。
    VOLEitH検証プロトコルは、主に以下のような具体的な計算処理で構成される:
    \begin{itemize}
        \item \textbf{コミットメントの検証}: 証明者が行ったコミットメントが正しく開示されているか、またはあるプロパティを満たしているかの検証。
        \item \textbf{ハッシュ関数の計算}: チャレンジ生成やメッセージダイジェストの一致性確認のために、KeccakやSHAなどのハッシュ関数を再計算し、その出力を比較する。
        \item \textbf{VOLE相関の検証}: VOLEプロトコルによって生成された相関が、その数学的特性(例:線形性)を満たしているかの検証。これには有限体上の加算、乗算などの算術演算が含まれる。
        \item \textbf{ランダム性の検証}: プロトコル中に使用されたランダムネスが正しく利用されているかの確認。
    \end{itemize}
    これらの検証ロジックを、SNARKが扱うことができる形式であるR1CS(Rank-1 Constraint System)の算術回路として表現する。このR1CS化のプロセスでは、VOLEitH検証プロトコルを構成する多数の論理ゲートや算術演算を、SNARKが解釈可能な制約形式に変換する必要があり、回路の効率的な設計がSNARK証明生成のパフォーマンスに大きく影響する。
    この「検証回路」は、\(\pi_{\text{VOLEitH}}\)を入力として受け取り、それが有効であれば真を、無効であれば偽を出力する。
    \item \textbf{ステップ3:SNARK証明の生成}
    証明者は、ステップ2で構築した「VOLEitH検証回路」に対して、Groth16プロトコルを用いて証明\(\pi_{\text{SNARK}}\)を生成する。
    この\(\pi_{\text{SNARK}}\)は、「ある有効な\(\pi_{\text{VOLEitH}}\)を知っており、その証明に対する検証計算が正しく実行されたこと」を簡潔に証明するものである。
    Groth16の重要な特性は、証明される計算の複雑さ(回路のゲート数)に関わらず、生成される\(\pi_{\text{SNARK}}\)のサイズが常に一定(約1KB程度)である点にある。この固定された小さな証明サイズは、ブロックチェーン上でのデータ保存コストを最小化し、検証の効率性を最大限に高める上で極めて有利である。
    \item \textbf{ステップ4:オンチェーン検証}
    最終的に、コンパクトなSNARK証明\(\pi_{\text{SNARK}}\)と、計算の公開入力のみがEthereum上の検証者スマートコントラクトに送信される。
    スマートコントラクトは、事前にデプロイされた検証鍵を用いて\(\pi_{\text{SNARK}}\)を検証する。この検証は、数回のペアリング演算で完了するため、極めて少ないガス代で実行可能である。
\end{enumerate}

この一連のプロセスにより、クライアント側では証明生成の大部分を軽量なVOLEitHプロトコルが担い、ブロックチェーン側では検証コストの高い計算がコンパクトなSNARK証明の検証に置き換えられる。
本研究では、この革新的なハイブリッドアーキテクチャをRust言語で実装した。VOLEitHの実装には\texttt{schmivitz}ライブラリを、SNARK回路の構築と証明生成には\texttt{arkworks}エコシステムを利用し、オンチェーン検証用のSolidityコントラクトは\texttt{ark-groth16}の機能を用いて生成した。これにより、理論的に提唱されてきたVOLEitHとSNARKの連携による効率化を、具体的なシステムとして構築・評価する初の試みの一つとして、その実用性とパフォーマンスを定量的に明らかにする。本研究の成果は、リソース制約の厳しい環境下でのゼロ知識証明の適用範囲を大きく広げるものである。

\subsection{評価対象回路}
本ベンチマークでは、プロトコルの基本的な性能と、より実践的な応用における性能の両方を評価するため、2種類の回路セットを用いた。
\begin{itemize}
    \item \textbf{SHA256回路}:
    \begin{itemize}
        \item 内容: \textbf{SHA-256} の評価回路。これは、特定の入力値に対するSHA-256ハッシュ計算を実行する論理回路として表現される。具体的には、ハッシュ関数の各ステップ(パディング、メッセージブロックの生成、圧縮関数の適用など)が算術回路のゲートとしてモデル化されており、入力データから最終的なハッシュ値が出力されるまでの計算パス全体を検証可能としている。SHA-256の計算はビット単位の操作が多く含まれるため、特にANDゲートを多用する構造となっており、これがSNARK回路に変換する際に多くの制約数を生成する要因となる。この回路は、暗号技術で広く利用される標準的なハッシュ関数であり、複雑な計算の代表例として用いた。
        \item 形式: これらの回路は、\href{https://github.com/GaloisInc/swanky/tree/dev/bristol-fashion/circuits}{Bristol Fashion}形式で記述されたものを、本研究で利用するVOLEitHライブラリに適した形式に変換して使用した。
        \item 目的: VOLEitHプロトコル単体の性能と、既存のZKP実装(Circom)との比較評価に用いることで、複雑なハッシュ計算における本ハイブリッドアプローチの性能特性を検証する。
    \end{itemize}
    \item \textbf{E2E評価用基本回路}:
    \begin{itemize}
        \item 内容: \textbf{100ゲート}および\textbf{1000ゲート}の\textbf{ADD(加算)回路}と\textbf{AND(乗算)回路}。これらの単純な算術回路は、ゼロ知識証明システムにおける基本的な演算の性能を評価するために広く用いられる。
        \item 目的: エンドツーエンド(E2E)の性能評価において、回路の規模(ゲート数)と種類(加算/乗算)が、VOLEitHフェーズとSNARKフェーズの各メトリクスにどのような影響を与えるかを詳細に分析するために用いる。特に、加算ゲートと乗算ゲートはZKPプロトコルにおいて異なる計算コストを持つことが多いため、それぞれのゲートタイプに特化した評価を行うことで、本ハイブリッドアーキテクチャのボトルネックや最適化の機会を特定するための重要な知見を得ることを目指す。
    \end{itemize}
\end{itemize}

\subsection{測定項目}
本研究では、証明システムの性能と実用性を多角的に評価するため、以下のメトリクスを測定対象とした。
\begin{table}[hbtp]
    \centering
    \begin{tabular}{|l|l|l|l|}
        \hline
        \textbf{メトリクス} & \textbf{説明} & \textbf{単位/備考} & \textbf{測定の意義} \\
        \hline
        証明生成時間 & 証明者が、ある計算に対する証明を生成するために要する時間 & ms または $\mu$s & クライアントサイドの性能、ユーザー体験への影響 \\
        \hline
        証明検証時間 & 検証者が、与えられた証明の正当性を検証するために要する時間 & ms または $\mu$s & 証明の正当性確認に要する計算量 \\
        \hline
        証明サイズ & 生成された証明データの大きさ & バイト(B) & オンチェーンでのデータ保存コスト、通信量 \\
        \hline
        通信オーバーヘッド & 非対話型証明において、証明者が検証者に送信する必要がある総データ量。基本的に証明サイズとほぼ同等 & バイト(B) & ネットワーク負荷、証明の運搬コスト \\
        \hline
        計算負荷 & 証明生成および検証プロセス中に消費されるCPU使用率および最大メモリ使用量 & CPU使用率(\%)、メモリ(MB) & リソース消費、ハードウェア要件の評価 \\
        \hline
        SNARK制約数 & VOLEitHの検証ロジックをSNARK回路に変換した際に生成されるR1CSの制約数 & - & SNARK証明生成時間とオンチェーン検証ガス代に大きく影響 \\
        \hline
        オンチェーン検証ガス代 & 生成されたSNARK証明をEthereumのスマートコントラクトで検証する際に消費されるガス量 & gas & ブロックチェーン上での運用コスト、経済合理性 \\
        \hline
    \end{tabular}
    \caption{ベンチマーク測定項目一覧}\label{tab:bench_metrics}
\end{table}

\subsection{評価環境}
すべてのベンチマークは、以下の統一された環境で実施した。
\begin{itemize}
    \item \textbf{ハードウェア}:
    \begin{itemize}
        \item CPU: Apple M1
        \item メモリ: 16GB
    \end{itemize}
    \item \textbf{ソフトウェア}:
    \begin{itemize}
        \item 言語: Rust
        \item VOLEitH実装: \texttt{schmivitz} ライブラリ
    \end{itemize}
\end{itemize}

