\section{結果と分析 (Results and Analysis)}
\label{sec:results_analysis}
本章では、設計したベンチマークに基づき、VOLEitHとSNARKを組み合わせたハイブリッドアーキテクチャの性能を多角的に評価する。まず4.1節でVOLEitHプロトコル単体の性能を評価し、その基本的な特性(特に証明サイズと生成・検証速度)を明らかにする。
次に4.2節で、VOLEitHの証明をSNARKで圧縮しオンチェーン検証するまでのプロセス全体の性能を評価する。最後に4.5節でこれらの結果を統合し本アーキテクチャ全体の有効性とトレードオフについて考察し、4.6節で実用アプリケーションへの適用可能性について論じる。

\subsection{VOLEitHフェーズおよびSNARKフェーズの性能評価}
前節(4.1節)で、VOLEitHプロトコル単体では基本的なADD/AND回路においても証明サイズが依然として大きく、そのままではオンチェーン検証に適さないという課題が明らかになった。
この課題を解決するため、本節ではVOLEitHの証明をSNARKで圧縮しオンチェーン検証するまでのVOLEitHフェーズおよびSNARKフェーズを合わせたプロセス全体の性能を評価する。
ベンチマークは、100ゲートおよび1000ゲートのADD回路とAND回路を用いて実施した。

まず、VOLEitHフェーズの性能を\textbf{表\ref{tab:e2e_vole_phase}}に示す。

\begin{table}[htbp]
    \centering
    \caption{VOLEitHフェーズおよびSNARKフェーズのベンチマーク - VOLEフェーズの性能}
    \label{tab:vole_phase}
    \begin{tabular}{|l|l|l|l|l|}
        \hline
        \textbf{メトリクス} & \textbf{100 add} & \textbf{100 and} & \textbf{1000 add} & \textbf{1000 and} \\
        \hline
        証明生成時間 & 279.012~\ensuremath{\mu\mathrm{s}} & 476.5~\ensuremath{\mu\mathrm{s}} & 790.062~\ensuremath{\mu\mathrm{s}} & 1.649~ms \\
        証明検証時間 & 68.75~\ensuremath{\mu\mathrm{s}} & 274.566~\ensuremath{\mu\mathrm{s}} & 585.6~\ensuremath{\mu\mathrm{s}} & 1.082~ms \\
        証明サイズ & 21,361 B & 42,491 B & 21,319 B & 233,175 B \\
        通信オーバーヘッド & 21,426 B & 42,556 B & 21,384 B & 233,240 B \\
        \hline
    \end{tabular}
\end{table}

表\ref{tab:e2e_vole_phase}から、VOLEフェーズにおいては、回路のゲート数が増加するにつれて、証明生成時間、検証時間、証明サイズ、通信オーバーヘッドが増加することがわかる。
特に、ANDゲート回路はADDゲート回路と比較して、同程度のゲート数であっても証明生成時間、検証時間、証明サイズが大幅に増加する傾向にある。
これは、\texttt{soft\_spoken}の実装において、ANDゲートのような乗算処理がADDゲートのような加算処理よりも多くのVOLEプロトコルラウンドや計算を必要とすることに起因すると考えられる。

次に、SNARKフェーズの性能を\textbf{表\ref{tab:e2e_snark_phase}}に示す。このフェーズでは、VOLEitHの証明をSNARK(Groth16)形式に変換し、オンチェーン検証に適した形に圧縮する。

\begin{table}[htbp]
    \centering
    \caption{VOLEitHフェーズおよびSNARKフェーズのベンチマーク - SNARKフェーズの性能}
    \label{tab:snark_phase}
    \begin{tabular}{|l|l|l|l|l|}
        \hline
        \textbf{メトリクス} & \textbf{100 add} & \textbf{100 and} & \textbf{1000 add} & \textbf{1000 and} \\
        \hline
        証明生成時間 & 272 ms & 1,794 ms & 324 ms & 8,003 ms \\
        制約数 & 86,080 & 3,471,680 & 86,080 & 33,942,080 \\
        証明サイズ & 1,055 B & 1,055 B & 1,055 B & 1,055 B \\
        ガス消費量 & 208,967 & 208,967 & 208,967 & 208,967 \\
        \hline
    \end{tabular}
\end{table}

表\ref{tab:e2e_snark_phase}から、SNARKフェーズではVOLEフェーズとは異なる特性が明らかになる。
最も注目すべきは、最終的なSNARK証明のサイズが、回路のゲート数や種類に関わらず\textbf{1,055バイト}に固定されている点である。
また、オンチェーン検証のガス代も\textbf{208,967 gas}で一定であり、これはSNARKの検証が固定コストで行われることを示している。
これにより、前節で課題となったVOLEitHの巨大な証明サイズが大幅に圧縮され、オンチェーン検証の実現可能性が飛躍的に向上する。

一方で、SNARK証明の生成時間と制約数には、回路の複雑性が大きく影響している。特に、ANDゲート回路はADDゲート回路と比較して、制約数が大幅に増加し、それに伴い証明生成時間も急増している。
例えば、1000 ANDゲート回路では、制約数が33,942,080に達し、証明生成に8,003 ms(約8秒)を要している。

この関係性をより視覚的に示すため、\textbf{図1}にSNARKの制約数と証明生成時間の関係を示す。

\begin{figure}[htbp]
    \centering
    % ここにSNARKの制約数と証明生成時間の関係を示すグラフを挿入
    \caption{SNARKの制約数と証明生成時間の関係}
    \label{fig:snark_constraints_time}
\end{figure}

図\ref{fig:snark_constraints_time}は、SNARKの証明生成時間が、回路の制約数、特に乗算ゲートに起因する制約数の増加に強く相関していることを明確に示している。
これは、SNARKの証明生成における主要な計算ボトルネックが、回路の複雑性、特に乗算の多さに起因することを示唆している。

\paragraph{主な観測事項}
本章で得られたVOLEitHフェーズおよびSNARKフェーズの測定結果から、以下の特徴が明らかになった。
\begin{itemize}
    \item ANDゲートはADDゲートよりも大幅に制約数と証明時間を増加させ、VOLEフェーズでも証明サイズを押し上げる。
    \item ADDのみの回路では制約数がほぼ一定であるのに対し、ANDゲート数に比例してSNARK制約が増える。
    \item SNARKフェーズの証明生成時間が、VOLEフェーズの生成・検証時間を大きく上回り、全体のボトルネックとなる。
    \item SNARK証明サイズおよびオンチェーン検証ガスは1,055バイトと約209k gasで一定であり、回路規模に依存しない。
    \item 総証明時間はSNARKフェーズの制約増加に強く影響されるため、複雑な回路ではクライアントデバイスでの実行が難しくなる。
\end{itemize}
