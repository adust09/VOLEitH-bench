\section{序論}
\subsection{ブロックチェーンZKPにおける二つの課題:オンチェーン検証とクライアントサイドプルービング}
ゼロ知識証明は、ある計算が正しく実行されたことを、その計算に関する情報を一切明らかにすることなく証明する暗号学的プロトコルである。
このプライバシー保護の性質は、ブロックチェーン技術の文脈で特に重要視される。
しかし、ZKPを実世界のアプリケーション、特にエンドユーザーが直接利用する分散型アプリケーションに導入するには、二つの大きな課題が存在する。

第一の課題は、\textbf{オンチェーン検証の制約}である。スマートコントラクト上で証明を検証する際には、証明サイズ、検証計算量、そしてそれに伴うガス代という、ブロックチェーン特有の厳しい制約に直面する。
証明が大きすぎたり、検証が複雑すぎたりすると、コストが現実的でなくなり、実用性が失われる。

第二の課題は、\textbf{クライアントサイドプルービングの要求}である。アプリケーションが広く普及するためには、エンドユーザーが自身のデバイス(スマートフォンやPCなど)で証明を生成できなければならない。
しかし、多くのzk-SNARKsは検証の軽量性と引き換えに、証明生成に膨大な計算リソースを要求する。この「証明生成の重さ」はユーザー体験を著しく損ない、技術の普及を妨げる大きな障壁となっている。

したがって、理想的な証明システムは、これら二つの課題を同時に解決する必要がある。すなわち、クライアントにとっては\textbf{軽量な証明生成}を、ブロックチェーンにとっては\textbf{効率的なオンチェーン検証}を両立させなければならない。
VOLE-in-the-Head(VOLEitH)のような対称鍵暗号ベースの証明システムは、特に証明生成の軽量性に優れていることから、このクライアントサイドプルービングの課題に対する有望な解決策として注目されている。

\subsection{VOLEベースZKPとVOLE-in-the-Head}
この課題に対し、証明者の計算効率を大幅に向上させる新しいZKPの系統として、VOLE(Vector Oblivious Linear Evaluation)ベースのプロトコルが登場した。
これらのプロトコルは、従来のSNARKs(Succinct Non-Interactive Argument of Knowledge)で主流であったR1CS(Rank-1 Constraint System)とは異なるアプローチを取り、特に証明者の計算負荷を軽減することに成功している。
SNARKsは証明サイズが小さく検証が高速なためオンチェーン検証で広く利用されており、Groth16\cite{groth16}やPLONK\cite{plonk}のような多くの証明システムが提案されている。

その中でもVOLE-in-the-Head (VOLE itH) は、VOLEベースの対話型プロトコルにFiat-Shamir変換を適用することで、誰でも検証可能な公開証明(publicly verifiable proof)を生成可能にした画期的な手法である\cite{voleith}。
これにより、証明者側の高い計算効率という利点を維持しつつ、オンチェーン検証への道が拓かれた。

\subsection{研究の目的と貢献}
VOLEitHは理論的には有望であるものの、その実用性、特にオンチェーン検証における具体的な性能やコストについては、まだ十分に明らかにされていない。
本研究の目的は、VOLEitHの特性を活かした軽量な証明者がEthereum上で検証することが可能であるか検証することにある。

具体的には、以下の項目を詳細に測定・分析する。
\begin{itemize}
    \item 証明生成と検証にかかる時間
    \item 生成される証明のサイズ
    \item 証明者と検証者の計算負荷(CPU、メモリ)
    \item 最終的なオンチェーン検証にかかるガス代
\end{itemize}
本研究は、VOLEitHのオンチェーン応用における実現可能性と技術的なトレードオフを明らかにすることで、将来的なZKPシステムの設計と最適化、そしてブロックチェーンアプリケーションにおけるプライバシーとスケーラビリティの向上に貢献することを目指す。
