\section{ベンチマーク設計 (Benchmark Design)}
本章では、VOLEitHとSNARKを組み合わせたアーキテクチャの性能を定量的に評価するために設計したベンチマークの詳細について述べる。

\subsection{測定項目}
本研究では、証明システムの性能と実用性を多角的に評価するため、以下のメトリクスを測定対象とした。
\begin{table}[hbtp]
    \centering
    \begin{tabular}{|l|l|l|}
        \hline
        \textbf{メトリクス} & \textbf{説明} & \textbf{単位/備考} \\
        \hline
        証明生成時間 (Proof Generation Time) & 証明者が、ある計算に対する証明を生成するために要する時間 & ms または $\mu$s \\
        \hline
        証明検証時間 (Proof Verification Time) & 検証者が、与えられた証明の正当性を検証するために要する時間 & ms または $\mu$s \\
        \hline
        証明サイズ (Proof Size) & 生成された証明データの大きさ & バイト(B) \\
        \hline
        通信オーバーヘッド (Communication Overhead) & 非対話型証明において、証明者が検証者に送信する必要がある総データ量。基本的に証明サイズとほぼ同等 & バイト(B) \\
        \hline
        計算負荷 (Computation Load) & 証明生成および検証プロセス中に消費されるCPU使用率および最大メモリ使用量 & CPU使用率(\%)、メモリ(MB) \\
        \hline
        SNARK制約数 (Number of constraints) & VOLEitHの証明をSNARKに変換する際に生成されるR1CSの制約数。証明生成時間に大きく影響 & - \\
        \hline
        オンチェーン検証ガス代 (On-Chain Verification Gas Cost) & 生成されたSNARK証明をEthereumのスマートコントラクトで検証する際に消費されるガス量 & gas \\
        \hline
    \end{tabular}
    \caption{ベンチマーク測定項目一覧}
\end{table}

\subsection{評価環境}
すべてのベンチマークは、以下の統一された環境で実施した。
\begin{itemize}
    \item \textbf{ハードウェア}:
    \begin{itemize}
        \item CPU: Apple M1
        \item メモリ: 16GB
    \end{itemize}
    \item \textbf{ソフトウェア}:
    \begin{itemize}
        \item 言語: Rust
        \item ベンチマークツール: \texttt{cargo bench}
        \item VOLEitH実装: \texttt{soft\_spoken} ライブラリ
        \item スマートコントラクト開発・テスト: Foundryフレームワーク
        \item Solidityバージョン: 0.8.20
    \end{itemize}
\end{itemize}

\subsection{評価対象回路}
本ベンチマークでは、プロトコルの基本的な性能と、より実践的な応用における性能の両方を評価するため、2種類の回路セットを用いた。
\begin{itemize}
    \item \textbf{SHA256回路}:
    \begin{itemize}
        \item 内容: \textbf{SHA-256} 。これらは暗号技術で広く利用される標準的なハッシュ関数であり、複雑な計算の代表例として用いた。
        \item 形式: これらの回路は、\href{https://github.com/GaloisInc/swanky/tree/dev/bristol-fashion/circuits}{Bristol Fashion}形式で記述されたものを、本研究で利用するVOLEitHライブラリに適した形式に変換して使用した。
        \item 目的: VOLEitHプロトコル単体の性能と、既存のZKP実装(Circom)との比較評価に用いる。
    \end{itemize}
    \item \textbf{E2E評価用基本回路}:
    \begin{itemize}
        \item 内容: \textbf{100ゲート}および\textbf{1000ゲート}の\textbf{ADD(加算)回路}と\textbf{AND(乗算)回路}。
        \item 目的: エンドツーエンド(E2E)の性能評価、特に回路の規模(ゲート数)と種類(加算/乗算)が、VOLEitHフェーズとSNARKフェーズの各メトリクスにどのような影響を与えるかを詳細に分析するために用いる。
    \end{itemize}
\end{itemize}
