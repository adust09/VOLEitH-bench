\documentclass[a4paper, 11pt]{article}

% 日本語対応
\usepackage{amsmath}
\usepackage{amssymb}
\usepackage{amsfonts}
\usepackage{graphicx} % 図の挿入
\usepackage[hidelinks]{hyperref} % リンク(枠・色を非表示)
\usepackage{url} % URL表示
\usepackage{enumitem} % リストのカスタマイズ
\usepackage{float} % 図表の配置制御

% 日本語環境の設定 (必要に応じて変更)
\usepackage[ipaex]{zxjafont} % XeLaTeX/LuaLaTeXの場合
% \usepackage[utf8]{inputenc} % pdfLaTeXの場合
% \usepackage[T1]{fontenc} % pdfLaTeXの場合
% \usepackage[japanese]{babel} % pdfLaTeXの場合

% タイトル、著者情報
\title{VOLE-in-the-Headゼロ知識証明のオンチェーン検証における実現可能性とコスト分析}
\author{著者名 \\ 所属機関}
\date{\today}

\begin{document}

\maketitle

\begin{abstract}
% アブストラクトのプレースホルダー
本研究では、効率的な証明者計算を特徴とするVOLE-in-the-Head(VOLEitH)ゼロ知識証明の、パブリックブロックチェーン上での実用性を評価した。特に、オンチェーン検証に焦点を当て、証明生成から検証までのエンドツーエンド(E2E)プロセスにおける計算コスト、証明サイズ、そしてEthereum上でのガス代を詳細に測定・分析した。ベンチマークとして、標準的な暗号学的ハッシュ関数(SHA-256, Keccak-F)および基本的な論理ゲート回路を用いた。結果として、VOLEitHは既存のZKP実装(Circom)と比較して証明生成が最大15.5倍高速である一方、証明サイズが6000倍以上増大することが確認された。さらに、VOLEitHの証明をSNARKで圧縮することにより、証明サイズを1,055バイトに固定し、オンチェーン検証のガス代を約21万ガスに抑えられることを示した。本稿は、VOLEitHをオンチェーンアプリケーションへ適用する際の技術的なトレードオフを明らかにし、将来的なスケーラビリティ向上のための示唆を与える。
\end{abstract}

\newpage

% 各セクションの読み込み
\section{序論 (Introduction)}
\subsection{ゼロ知識証明の進化とオンチェーン検証の課題}
ゼロ知識証明(Zero-Knowledge Proof, ZKP)は、ある計算が正しく実行されたことを、その計算に関する入力情報(witness)を一切明らかにすることなく検証者に証明するための暗号学的プロトコルである。この「ゼロ知識」という性質は、プライバシー保護が強く求められる現代のデジタル社会において極めて重要であり、認証、電子投票、そして特にブロックチェーン技術の分野でその応用が期待されている。

ブロックチェーン、特にスマートコントラクトプラットフォームにおいては、計算の正当性をトラストレストランザクションとしてオンチェーンで検証する能力が、スケーラビリティとプライバシーの両方を向上させる鍵となる。例えば、計算量の多い処理をオフチェーンで実行し、その結果の正当性のみをZKPを用いてオンチェーンで検証する「ZKロールアップ」は、Ethereumのスケーラビリティ問題を解決する主要なアプローチとして注目を集めている。しかし、ZKPをオンチェーンで検証する際には、証明サイズ、検証計算量、そしてそれに伴うガス代という、ブロックチェーン特有の厳しい制約が大きな課題となる。

\subsection{VOLEベースZKPとVOLE-in-the-Head}
この課題に対し、証明者(Prover)の計算効率を大幅に向上させる新しいZKPの系統として、VOLE(Vector Oblivious Linear Evaluation)ベースのプロトコルが登場した。これらのプロトコルは、従来のSNARKsで主流であったR1CS(Rank-1 Constraint System)とは異なるアプローチを取り、特に証明者の計算負荷を軽減することに成功している。

その中でもVOLE-in-the-Head (VOLEitH) は、VOLEベースの対話型プロトコルにFiat-Shamir変換を適用することで、誰でも検証可能な公開証明(publicly verifiable proof)を生成可能にした画期的な手法である。これにより、証明者側の高い計算効率という利点を維持しつつ、オンチェーン検証への道が拓かれた。

\subsection{研究の目的と貢献}
VOLEitHは理論的には有望であるものの、その実用性、特にオンチェーン検証における具体的な性能やコストについては、まだ十分に明らかにされていない。本研究の目的は、VOLEitHをSNARKと組み合わせたハイブリッドアーキテクチャを構築し、そのエンドツーエンドの性能を定量的に評価することにある。

具体的には、以下の項目を詳細に測定・分析する。
\begin{itemize}
    \item 証明生成と検証にかかる時間
    \item 各フェーズで生成される証明のサイズ
    \item 証明者と検証者の計算負荷(CPU、メモリ)
    \item 最終的なオンチェーン検証にかかるガス代
\end{itemize}
本研究は、VOLEitHのオンチェーン応用における実現可能性と技術的なトレードオフを明らかにすることで、将来的なZKPシステムの設計と最適化、そしてブロックチェーンアプリケーションにおけるプライバシーとスケーラビリティの向上に貢献することを目指す。

\subsection{論文の構成}
本論文は以下のように構成される。第2章では、本研究の基礎となるZKP、VOLE、VOLEitH、SNARKsの技術的背景を解説する。第3章では、性能評価に用いたベンチマークの設計と環境について詳述する。第4章では、ベンチマーク結果を提示し、詳細な分析と考察を行う。第5章では、関連研究との比較を行う。最後に第6章で、本研究の結論と今後の展望を述べる。

\section{背景 (Background)}
本章では、我々の研究の基礎となる暗号学的概念とプロトコルについて解説する。

\subsection{ゼロ知識証明の基礎}
ゼロ知識証明(ZKP)は、証明者(Prover)が検証者(Verifier)に対し、ある表明が真であることを、その表明の真実性を担保する情報(witness)を一切明かすことなく納得させるためのプロトコルである。ZKPは以下の3つの性質を満たす必要がある。
\begin{enumerate}
    \item \textbf{完全性 (Completeness)}: 証明者の表明が真であるならば、正直な証明者は正直な検証者を必ず納得させることができる。
    \item \textbf{健全性 (Soundness)}: 証明者の表明が偽であるならば、不正な証明者が正直な検証者を騙して納得させられる確率は、無視できるほど小さい。
    \item \textbf{ゼロ知識性 (Zero-Knowledge)}: 検証者は、証明の正当性以外には、証明者の持つ秘密情報について何も知ることができない。
\end{enumerate}
ZKPは、証明者と検証者の間で複数回のやり取りを必要とする「対話型証明システム」と、証明者が一度証明を公開すれば誰でも検証できる「非対話型証明システム(NIZK)」に大別される。オンチェーン検証のように、不特定多数の検証者が非同期に検証を行う環境では、NIZKが不可欠となる。

\subsection{VOLEと関連プロトコル}
VOLE(Vector Oblivious Linear Evaluation)は、二者間(SenderとReceiver)のセキュアな計算プロトコルであり、近年のZKPシステムの効率化に大きく貢献している。基本的なVOLEプロトコルでは、Senderが持つアフィン変換 \texttt{f(x) = ax + b} と、Receiverが持つベクトル \texttt{u} に対し、Receiverが \texttt{v = au + b} を計算する。この過程で、Senderは \texttt{u} について、Receiverは \texttt{a, b} について何も知ることができない。

VOLEベースのプロトコルの多くは、LPN(Learning Parity with Noise)仮定の困難性に基づいている。LPN仮定とは、ランダムな線形方程式系にノイズが加わったものから、元の線形関係を復元することが計算量的に困難であるという仮定であり、これにより量子コンピュータに対しても耐性を持つ(ポスト量子暗号)と考えられている。

VOLEをZKPに応用するために、SPVOLE(Single-Point VOLE)やZP-VOLE(Zero-Point VOLE)といった派生プロトコルが考案された。これらは、特定の点でのみ値が非ゼロになるようなベクトルを効率的に扱うためのプロトコルであり、算術回路の各ゲートにおける制約を表現するのに適している。

\subsection{VOLE-in-the-Head (VOLEitH)}
VOLE-in-the-Head (VOLEitH) は、VOLEベースのプロトコルをZKPに昇華させた手法である。その基本的なアイデアは、証明したい算術回路のワイヤ値をProverがコミットし、Verifierがランダムな線形結合をチェックすることで、回路全体の計算が正しく行われたかを検証するというものである。

本研究で利用する\texttt{soft\_spoken}ライブラリは、VOLEitHの具体的な実装の一つである。\texttt{soft\_spoken}では、SPVOLEとZP-VOLEを巧みに組み合わせることで、効率的な証明生成を実現している。プロトコルの流れは以下のようになる。
\begin{enumerate}
    \item \textbf{コミットメント}: Proverは、算術回路の各ワイヤの値を表すベクトルにコミットする。
    \item \textbf{チャレンジ}: Verifierは、ランダムなチャレンジ(乱数)をProverに送信する。
    \item \textbf{レスポンス}: Proverは、チャレンジに基づき、コミットしたベクトル間の線形関係が成立することを示すための情報を計算し、Verifierに返信する。
    \item \textbf{検証}: Verifierは、Proverからのレスポンスと自身が持つ情報を用いて、線形関係が成立するかをチェックする。もし成立すれば、Proverが回路を正しく計算したと確信する。
\end{enumerate}
この対話型のプロトコルを非対話的にするため、Fiat-Shamir変換が用いられる。これは、Verifierが生成するランダムなチャレンジを、プロトコルのここまでの情報(コミットメント等)をハッシュ関数に入力することで自己生成するテクニックである。これにより、ProverはVerifierとの対話なしに証明を一方的に生成でき、生成された証明は誰でも検証可能なNIZKとなる。

\subsection{SNARKsとオンチェーン検証}
VOLEitHによって生成された証明は、証明者の計算効率は高いものの、証明サイズが非常に大きいという問題がある(\ref{sec:results_analysis} 4.1節で詳述)。このままではオンチェーン検証は現実的ではないため、証明をさらに圧縮する技術が必要となる。

ここで登場するのがSNARK(Succinct Non-interactive Argument of Knowledge)である。SNARK、特に現在主流のGroth16などは、証明サイズを数百バイト程度まで劇的に圧縮することができる。これは、計算の正当性の問題を、特定の性質を持つ多項式の存在問題に変換し、その多項式に対するコミットメントをペアリングという暗号学的道具立てを用いて効率的に検証することで実現される。

SNARKの検証は、Ethereum Virtual Machine (EVM) 上でスマートコントラクトとして実装できる。検証コントラクトは、いくつかのペアリング演算と比較を行うだけで証明を検証できるため、計算量が比較的小さく、ガス代を低く抑えることが可能である。これにより、VOLEitHの巨大な証明をSNARKで圧縮し、そのコンパクトなSNARK証明をオンチェーンで検証するという、本研究のハイブリッドアーキテクチャが実現される。
\section{実現可能性分析と主要な知見}
Milestone 1と2では、VOLEitHの証明をオンチェーンで扱うために複数の手法を検討し、最終的にGroth16によるSNARK Wrappingを実装・評価した。本節では、その意思決定過程と主要な洞察を整理する。

\subsection{SNARK Wrapping (Groth16)}
最も直接的なアプローチは、VOLEitHの検証ロジック全体をGroth16で包む方法である。この手法により、回路規模やゲート種別に依らず証明サイズを\textbf{1,055バイト}、オンチェーン検証ガスを\textbf{208,967 gas}に固定できた。一方で、制約数は
\begin{align*}
16,640 \times n + 2,113,664
\end{align*}
と線形に増加し、特に非線形ゲートを多く含む回路では証明生成時間が急増する。実装は以下のコンポーネントで構成される。
\begin{itemize}
    \item \texttt{schmivitz-snark}: VOLEitH検証ロジックをarkworksベースのGroth16で証明するためのラッパー
    \item \texttt{VOLEitH-bench}: Groth16圧縮後の証明をEVMで検証し、エンドツーエンド性能を測定するベンチマーク
\end{itemize}
SHA-256のようにANDゲートが2万個を超える回路では、この制約爆発によりSNARKフェーズの実装を断念せざるを得なかった。

\subsection{Foldingアプローチの検討}
Milestone 1では、VOLEitHの検証ロジックが階層的な構造を持つことから、NIVCを用いたFoldingスキームへの写像も検討した。特にGGM木の再構成部分はPRGとハッシュ関数のみから構成されるため、Foldingを用いれば証明サイズをより小さく保てる可能性がある。しかし、現状の\texttt{sonobe}実装はNIVCをサポートしておらず、SchmivitzのVOLE実装もGGM木の完全な再構成を含んでいないため、本プロジェクトでは採用を見送った。この方向性は今後の改良候補として残している。

\subsection{Blobを用いた格納案}
当初はEIP-4844のBlobにVOLEitH証明を格納する案も検討したが、現状のEVMはオンチェーンからBlobデータへアクセスできない。そのため、証明サイズが縮小されない限りオンチェーン検証に利用できないと判断し、この案も採用しなかった。

\section{ベンチマーク設計 (Benchmark Design)}
本章では、VOLEitHとSNARKを組み合わせたアーキテクチャの性能を定量的に評価するために設計したベンチマークの詳細について述べる。

\subsection{測定項目}
本研究では、証明システムの性能と実用性を多角的に評価するため、以下のメトリクスを測定対象とした。
\begin{table}[hbtp]
    \centering
    \begin{tabular}{|l|l|l|}
        \hline
        \textbf{メトリクス} & \textbf{説明} & \textbf{単位/備考} \\
        \hline
        証明生成時間 (Proof Generation Time) & 証明者が、ある計算に対する証明を生成するために要する時間 & ms または $\mu$s \\
        \hline
        証明検証時間 (Proof Verification Time) & 検証者が、与えられた証明の正当性を検証するために要する時間 & ms または $\mu$s \\
        \hline
        証明サイズ (Proof Size) & 生成された証明データの大きさ & バイト(B) \\
        \hline
        通信オーバーヘッド (Communication Overhead) & 非対話型証明において、証明者が検証者に送信する必要がある総データ量。基本的に証明サイズとほぼ同等 & バイト(B) \\
        \hline
        計算負荷 (Computation Load) & 証明生成および検証プロセス中に消費されるCPU使用率および最大メモリ使用量 & CPU使用率(\%)、メモリ(MB) \\
        \hline
        SNARK制約数 (Number of constraints) & VOLEitHの証明をSNARKに変換する際に生成されるR1CSの制約数。証明生成時間に大きく影響 & - \\
        \hline
        オンチェーン検証ガス代 (On-Chain Verification Gas Cost) & 生成されたSNARK証明をEthereumのスマートコントラクトで検証する際に消費されるガス量 & gas \\
        \hline
    \end{tabular}
    \caption{ベンチマーク測定項目一覧}
\end{table}

\subsection{評価環境}
すべてのベンチマークは、以下の統一された環境で実施した。
\begin{itemize}
    \item \textbf{ハードウェア}:
    \begin{itemize}
        \item CPU: Apple M1
        \item メモリ: 16GB
    \end{itemize}
    \item \textbf{ソフトウェア}:
    \begin{itemize}
        \item 言語: Rust
        \item ベンチマークツール: \texttt{cargo bench}
        \item VOLEitH実装: \texttt{soft\_spoken} ライブラリ
        \item スマートコントラクト開発・テスト: Foundryフレームワーク
        \item Solidityバージョン: 0.8.20
    \end{itemize}
\end{itemize}

\subsection{評価対象回路}
本ベンチマークでは、プロトコルの基本的な性能と、より実践的な応用における性能の両方を評価するため、2種類の回路セットを用いた。
\begin{itemize}
    \item \textbf{SHA256回路}:
    \begin{itemize}
        \item 内容: \textbf{SHA-256} 。これらは暗号技術で広く利用される標準的なハッシュ関数であり、複雑な計算の代表例として用いた。
        \item 形式: これらの回路は、\href{https://github.com/GaloisInc/swanky/tree/dev/bristol-fashion/circuits}{Bristol Fashion}形式で記述されたものを、本研究で利用するVOLEitHライブラリに適した形式に変換して使用した。
        \item 目的: VOLEitHプロトコル単体の性能と、既存のZKP実装(Circom)との比較評価に用いる。
    \end{itemize}
    \item \textbf{E2E評価用基本回路}:
    \begin{itemize}
        \item 内容: \textbf{100ゲート}および\textbf{1000ゲート}の\textbf{ADD(加算)回路}と\textbf{AND(乗算)回路}。
        \item 目的: エンドツーエンド(E2E)の性能評価、特に回路の規模(ゲート数)と種類(加算/乗算)が、VOLEitHフェーズとSNARKフェーズの各メトリクスにどのような影響を与えるかを詳細に分析するために用いる。
    \end{itemize}
\end{itemize}

\section{結果と分析 (Results and Analysis)}
\label{sec:results_analysis}
本章では、設計したベンチマークに基づき、VOLEitHの性能を多角的に評価する。まず4.1節でVOLEitHプロトコル単体の性能を評価し、既存実装との比較を通じてその基本的な特性を明らかにする。
次に4.2節で、VOLEitHの証明をSNARKで圧縮しオンチェーン検証するまでのエンドツーエンド(E2E)プロセス全体の性能を評価する。最後に4.3節で、これらの結果を統合し、本アーキテクチャ全体の有効性とトレードオフについて考察する。

\subsection{VOLEitH単体性能評価}
VOLEitHプロトコル自体の性能を評価するため、標準的な暗号学的ハッシュ関数であるSHA-256とKeccak-Fの回路を用いてベンチマークを実施した。
これらの回路はBristol Fashion形式で記述されたものを本研究用に変換したものである。

\textbf{表\ref{tab:voleith_standalone_benchmark}}に、Apple M1(メモリ16GB)環境で測定した両回路のベンチマーク結果を示す。

\begin{table}[htbp]
    \centering
    \caption{VOLEitH単体性能ベンチマーク (SHA-256 vs Keccak-F)}
    \label{tab:voleith_standalone_benchmark}
    \begin{tabular}{|l|l|l|}
        \hline
        \textbf{Metric} & \textbf{sha256} & \textbf{keccak\_f} \\
        \hline
        Proof Generation Time & 95 ms & 143 ms \\
        Proof Verification Time & 51 ms & 74 ms \\
        Proof Size & 4,927,342 B (\textasciitilde4.9 MB) & 8,416,569 B (\textasciitilde8.4 MB) \\
        Communication Overhead & 4,927,407 B (\textasciitilde4.9 MB) & 8,416,634 B (\textasciitilde8.4 MB) \\
        Prover Computation Load & 0.02\% CPU, 118.23 MB & 0.04\% CPU, 154.14 MB \\
        Verifier Computation Load & 0.04\% CPU, 138.89 MB & 0.04\% CPU, 158.1 MB \\
        \hline
    \end{tabular}
\end{table}

表\ref{tab:voleith_standalone_benchmark}から、回路の複雑性が性能に直接的な影響を与えることがわかる。
Keccak-FはSHA-256よりも複雑な回路構造を持つため、証明生成時間、検証時間、そして証明サイズのいずれにおいてもSHA-256を上回るコストが必要となった。
特筆すべきは証明サイズであり、SHA-256で約4.9MB、Keccak-Fでは約8.4MBにも達する。この巨大なデータサイズは、そのままではブロックチェーンのブロックサイズ制限やガス代の観点から、オンチェーンでの検証を著しく困難にする。

次に、VOLEitHの性能特性をより明確にするため、既存のZKP実装であるCircom(\cite{eprint:iacr:2023:681}より)によるSHA-256実装との比較を行う。
\textbf{表\ref{tab:sha256_comparison}}に両者の性能比較を示す。

\begin{table}[htbp]
    \centering
    \caption{SHA-256実装の性能比較 (VOLEitH vs Circom)}
    \label{tab:sha256_comparison}
    \begin{tabular}{|l|l|l|}
        \hline
        \textbf{実装} & \textbf{証明生成時間} & \textbf{証明サイズ} \\
        \hline
        \textbf{VOLEitH (本研究)} & \textbf{95 ms} & \textbf{\textasciitilde4.9 MB} \\
        Circom (先行研究 \cite{eprint:iacr:2023:681}) & \textasciitilde1,473 ms & \textasciitilde821 Bytes \\
        \hline
    \end{tabular}
\end{table}

表\ref{tab:sha256_comparison}は、VOLEitHの基本的なトレードオフを明確に示している。証明生成時間において、VOLEitHはCircom実装の約15.5倍高速である。
これは、証明者の計算効率を重視するVOLEベースのプロトコルの特性を強く反映している。
一方で、証明サイズに目を向けると、VOLEitHの証明は約4.9MBであるのに対し、Circom(Groth16)の証明は約821バイトと、VOLEitHが6000倍以上も大きい。

この結果から、VOLEitHはクライアントデバイスのような計算資源が限られた環境での高速な証明生成には適しているものの、生成された証明はオンチェーン検証には不向きであることがわかる。
この「証明は高速だが、証明自体が巨大」という課題が、本研究でSNARKによる証明圧縮アプローチを採用する動機となった。

\subsection{エンドツーエンド(E2E)性能評価}
前節でVOLEitH単体では証明サイズが大きすぎるという課題が明らかになったため、本節ではVOLEitHの証明をSNARKで圧縮し、オンチェーンで検証するまでのエンドツーエンド(E2E)プロセス全体の性能を評価する。
ベンチマークは、100ゲートおよび1000ゲートのADD回路とAND回路を用いて実施した。

まず、VOLEitHフェーズの性能を\textbf{表\ref{tab:e2e_vole_phase}}に示す。

\begin{table}[htbp]
    \centering
    \caption{E2Eベンチマーク - VOLEフェーズの性能}
    \label{tab:e2e_vole_phase}
    \begin{tabular}{|l|l|l|l|l|}
        \hline
        \textbf{Metric} & \textbf{100 add} & \textbf{100 and} & \textbf{1000 add} & \textbf{1000 and} \\
        \hline
        Proof Gen. Time & 279.012~\ensuremath{\mu\mathrm{s}} & 476.5~\ensuremath{\mu\mathrm{s}} & 790.062~\ensuremath{\mu\mathrm{s}} & 1.649~ms \\
        Proof Ver. Time & 68.75~\ensuremath{\mu\mathrm{s}} & 274.566~\ensuremath{\mu\mathrm{s}} & 585.6~\ensuremath{\mu\mathrm{s}} & 1.082~ms \\
        Proof Size & 21,361 B & 42,491 B & 21,319 B & 233,175 B \\
        Comm. Overhead & 21,426 B & 42,556 B & 21,384 B & 233,240 B \\
        \hline
    \end{tabular}
\end{table}

表\ref{tab:e2e_vole_phase}から、VOLEフェーズにおいては、回路のゲート数が増加するにつれて、証明生成時間、検証時間、証明サイズ、通信オーバーヘッドが増加することがわかる。
特に、ANDゲート回路はADDゲート回路と比較して、同程度のゲート数であっても証明生成時間、検証時間、証明サイズが大幅に増加する傾向にある。
これは、\texttt{soft\_spoken}の実装において、ANDゲートのような乗算処理がADDゲートのような加算処理よりも多くのVOLEプロトコルラウンドや計算を必要とすることに起因すると考えられる。

次に、SNARKフェーズの性能を\textbf{表\ref{tab:e2e_snark_phase}}に示す。このフェーズでは、VOLEitHの証明をSNARK(Groth16)形式に変換し、オンチェーン検証に適した形に圧縮する。

\begin{table}[htbp]
    \centering
    \caption{E2Eベンチマーク - SNARKフェーズの性能}
    \label{tab:e2e_snark_phase}
    \begin{tabular}{|l|l|l|l|l|}
        \hline
        \textbf{Metric} & \textbf{100 add} & \textbf{100 and} & \textbf{1000 add} & \textbf{1000 and} \\
        \hline
        Proof Gen. Time & 272 ms & 1,794 ms & 324 ms & 8,003 ms \\
        Constraints & 86,080 & 3,471,680 & 86,080 & 33,942,080 \\
        Proof Size & 1,055 B & 1,055 B & 1,055 B & 1,055 B \\
        Gas Cost & 208,967 & 208,967 & 208,967 & 208,967 \\
        \hline
    \end{tabular}
\end{table}

表\ref{tab:e2e_snark_phase}から、SNARKフェーズではVOLEフェーズとは異なる特性が明らかになる。
最も注目すべきは、最終的なSNARK証明のサイズが、回路のゲート数や種類に関わらず\textbf{1,055バイト}に固定されている点である。
また、オンチェーン検証のガス代も\textbf{208,967 gas}で一定であり、これはSNARKの検証が固定コストで行われることを示している。
これにより、前節で課題となったVOLEitHの巨大な証明サイズが大幅に圧縮され、オンチェーン検証の実現可能性が飛躍的に向上する。

一方で、SNARK証明の生成時間と制約数には、回路の複雑性が大きく影響している。特に、ANDゲート回路はADDゲート回路と比較して、制約数が大幅に増加し、それに伴い証明生成時間も急増している。
例えば、1000 ANDゲート回路では、制約数が33,942,080に達し、証明生成に8,003 ms(約8秒)を要している。

この関係性をより視覚的に示すため、\textbf{図1}にSNARKの制約数と証明生成時間の関係を示す。

\begin{figure}[htbp]
    \centering
    % ここにSNARKの制約数と証明生成時間の関係を示すグラフを挿入
    \caption{SNARKの制約数と証明生成時間の関係}
    \label{fig:snark_constraints_time}
\end{figure}

図\ref{fig:snark_constraints_time}は、SNARKの証明生成時間が、回路の制約数、特に乗算ゲートに起因する制約数の増加に強く相関していることを明確に示している。
これは、SNARKの証明生成における主要な計算ボトルネックが、回路の複雑性、特に乗算の多さに起因することを示唆している。

\paragraph{主な観測事項}
本章で得られたE2E測定結果から、以下の特徴が明らかになった。
\begin{itemize}
    \item ANDゲートはADDゲートよりも大幅に制約数と証明時間を増加させ、VOLEフェーズでも証明サイズを押し上げる。
    \item ADDのみの回路では制約数がほぼ一定であるのに対し、ANDゲート数に比例してSNARK制約が増える。
    \item SNARKフェーズの証明生成時間が、VOLEフェーズの生成・検証時間を大きく上回り、全体のボトルネックとなる。
    \item SNARK証明サイズおよびオンチェーン検証ガスは1,055バイトと約209k gasで一定であり、回路規模に依存しない。
    \item 総証明時間はSNARKフェーズの制約増加に強く影響されるため、複雑な回路ではクライアントデバイスでの実行が難しくなる。
\end{itemize}

\subsection{SNARK統合に関する洞察}
VOLEitHの証明をGroth16で包むと、証明サイズと検証コストは一定になる一方で、R1CS制約数が急増する。
ここでは制約数の内訳と、制約爆発の要因および緩和策を整理する。

\subsubsection{制約数の分解}
$n$ を拡張witnessの長さ(秘密入力数と乗算ゲート数の合計)とすると、全体の制約数は
\begin{align*}
16,640 \times n + 2,113,664
\end{align*}
と表せる。線形項に寄与するガジェットは表\ref{tab:linear_constraints}の通りであり、\texttt{compute\_validation\_aggregate}が支配的である。

\begin{table}[H]
    \centering
    \caption{線形に増加するガジェットの制約数}
    \label{tab:linear_constraints}
    \begin{tabular}{|l|r|}
        \hline
        ガジェット & 制約数 \\
        \hline
        $compute\_d\_delta$ & $128n$ \\
        $compute\_masked\_witness$ & $256n$ \\
        $compute\_validation\_aggregate$ & $16,512n$ \\
        \hline
        合計 & $\approx16,640n$ \\
        \hline
    \end{tabular}
\end{table}

また、回路サイズに依存しない定数項も無視できない(表\ref{tab:constant_constraints})。乗算ゲートが増えると線形項が支配するが、ベースラインとして約200万制約が常に必要になる。

\begin{table}[H]
    \centering
    \caption{定数項として加算されるガジェット}
    \label{tab:constant_constraints}
    \begin{tabular}{|l|r|}
        \hline
        ガジェット & 制約数 \\
        \hline
        $combine$ & $\sim2,097,152$ \\
        $compute\_actual\_validation$ & $\sim16,384$ \\
        最終整合性チェック & $\sim128$ \\
        \hline
        合計 & $\sim2,113,664$ \\
        \hline
    \end{tabular}
\end{table}

\subsubsection{Field Mappingがもたらす制約爆発}
SchmivitzにおけるVOLEitHは、$\mathbb{F}_2$、$\mathbb{F}_{2^8}$、$\mathbb{F}_{2^{64}}$、$\mathbb{F}_{2^{128}}$といった2進拡大体上で計算を行う。
一方で、Groth16のR1CSはBN254の素数体上で定義されるため、各ビット列をBoolean変数列に持ち上げる必要がある。
実装では以下のように、証明内の各値を逐一Boolean配列に射影している。

\begin{verbatim}
pub fn build_circuit(
    cs: ConstraintSystemRef<Bn254Fr>,
    proof: Proof<InsecureVole>,
) -> VoleVerificationBoolean {
    let witness_commitment_booleans: Vec<Vec<Boolean<Bn254Fr>>> = proof
        .witness_commitment
        .iter()
        .map(|value| f64b_to_boolean_array(cs.clone(), value).unwrap())
        .collect();

    let witness_challenges_booleans: Vec<Vec<Boolean<Bn254Fr>>> = proof
        .witness_challenges
        .iter()
        .map(|value| f128b_to_boolean_array(cs.clone(), value).unwrap())
        .collect();
    // ...
}
\end{verbatim}

この変換により、もともと単一の体要素で表現できた計算が数百ビットのAND/XORに展開され、制約数が爆発的に増加する。

\subsubsection{ANDゲートと\texttt{witness\_challenge}}
特にANDゲートを検証する際には、\texttt{witness\_challenge}と\texttt{masked\_witness}の全ビットについてANDおよびXOR演算を行い、部分積を合成する必要がある。
実装の核心は以下の通りであり、128ビット平方の積をBooleanレベルで計算するため、ANDゲート1つあたり$2^{14}$規模の制約が追加される。

\begin{verbatim}
for (i, challenge_bit) in challenge.iter().enumerate() {
    if i >= 128 { break; }
    for (j, masked_bit) in masked_witness.iter().enumerate() {
        if j >= 128 || i + j >= 128 { continue; }
        let and_result = Boolean::and(challenge_bit, masked_bit)?;
        product[i + j] = Boolean::xor(&product[i + j], &and_result)?;
    }
}
\end{verbatim}

ADDゲートでは\texttt{witness\_challenge}が不要なため制約数は一定だが、ANDゲートが増えるほど\texttt{compute\_validation\_aggregate}が繰り返し呼ばれ、SNARKフェーズ全体のボトルネックとなる。

\subsection{技術的ボトルネックと解決策}
上記の分析から、Field MappingとGGM木再構成が制約爆発の主要因であることが分かる。
本節では、これらを緩和するための具体的な研究方向を整理する。

\subsubsection{Field Mapping最適化とLookup Table}
Mystique\cite{eprint:2021:730}は、機械学習向けに$\mathbb{F}_2$と$\mathbb{F}_p$のデータ変換を効率化するVOLEベースZKであり、Lookup Table (LUT) を導入することでさらに高速化できることが最新研究\cite{eprint:2025:507}で示されている。
表\ref{tab:mystique_lut}に示す通り、LUTを用いた場合には実行時間が61--130倍短縮し、通信量も最大2.9倍削減できる。
VOLEitHのField MappingにMystique型LUTを適用できれば、SNARKフェーズの制約数削減に直結すると期待される。

\begin{table}[H]
    \centering
    \caption{MystiqueとLUT拡張の性能比較}
    \label{tab:mystique_lut}
    \begin{tabular}{|l|l|r|r|}
        \hline
        関数 & プロトコル & 実行時間 (s) & 通信量 (MB) \\
        \hline
        指数関数 & Mystique with LUT & 8.696 & 99.020 \\
                  & Mystique & 1130.020 & 291.435 \\
        除算 & Mystique with LUT & 9.837 & 110.684 \\
             & Mystique & 617.690 & 160.428 \\
        逆平方根 & Mystique with LUT & 11.836 & 147.903 \\
                 & Mystique & 824.639 & 212.211 \\
        \hline
    \end{tabular}
\end{table}

\subsubsection{GGM木最適化とFolding}
Schmivitzでは、VOLEitH検証で最もコストの高いGGM木再構成を簡略化しているが、SNARKで完全に検証する場合はこの部分が制約増大を引き起こす。
著者らはGGM木を効率化する手法\cite{eprint:2024:490}に加え、FAESTを改良したFAESTERを提案しており、署名サイズと計算量をともに改善している(表\ref{tab:faester})。
本研究で検討したFoldingスキームとこれらの最適化を組み合わせれば、将来的にGGM木再構成部の制約数を抑制できる。

\begin{table}[H]
    \centering
    \caption{FAESTとFAESTERの比較(セキュリティ128ビット)}
    \label{tab:faester}
    \begin{tabular}{|l|l|r|r|r|}
        \hline
        スキーム & バージョン & 署名サイズ (B) & 署名時間 (ms) & 検証時間 (ms) \\
        \hline
        FAEST & Slow & 50,063 & 4.3813 & 4.1023 \\
              & Fast & 63,363 & 0.4043 & 0.3953 \\
        FAESTER & Slow & 45,943 & 3.2823 & 4.4673 \\
                 & Fast & 60,523 & 0.4333 & 0.6103 \\
        \hline
    \end{tabular}
\end{table}

\subsection{総合考察とトレードオフ分析}
これまでの分析結果を統合し、VOLEitHとSNARKを組み合わせたアーキテクチャ全体の有効性とトレードオフについて考察する。

本研究で採用したアーキテクチャは、\textbf{図2}に示すように、証明者側(Prover)で2段階のプロセスを経て、検証者側(Verifier)であるブロックチェーン上で効率的な検証を実現するものである。

\begin{figure}[htbp]
    \centering
    % ここにVOLEitH + SNARKによる証明圧縮プロセスの概念図を挿入
    % 例: [Prover: 回路] -> [VOLEitH証明 (数MB)] -> [SNARK証明 (1KB)] -> [Verifier (オンチェーン)]
    \caption{VOLEitH + SNARKによる証明圧縮プロセスの概念図}
    \label{fig:proof_compression_flow}
\end{figure}

図\ref{fig:proof_compression_flow}が示す通り、本アーキテクチャは、VOLEitHが生成する巨大な証明(数MBオーダー)を、SNARKを用いてオンチェーン検証に適したコンパクトな証明(1KBオーダー)に圧縮する点に核心がある。
このアプローチにより、以下の2つの大きな利点を両立することが可能となる。

\begin{enumerate}
    \item \textbf{高速な証明者計算}: VOLEitHは、Circomのような従来のR1CSベースのシステムと比較して、証明者の計算が非常に高速である(表\ref{tab:sha256_comparison}参照)。
    これにより、計算能力が限られるクライアントデバイス(例: Webブラウザ、スマートフォン)でも、複雑な計算の証明を現実的な時間で生成できる可能性が広がる。
    \item \textbf{低コストなオンチェーン検証}: SNARK化された証明は、サイズが小さく、検証コストが回路の複雑さによらず一定であるため、ブロックチェーン上での検証コスト(ガス代)を大幅に削減し、予測可能なものにできる(表\ref{tab:e2e_snark_phase}参照)。
\end{enumerate}

一方で、このアーキテクチャには考慮すべきトレードオフも存在する。最大のトレードオフは、証明者側の計算負荷である。
証明者は、高速なVOLEitH証明生成に加えて、SNARK証明を生成するための追加の計算コストを負担する必要がある。
特に、回路が多くの乗算(ANDゲート)を含む場合、SNARKの制約数が急増し、SNARK証明の生成時間が全体のボトルネックとなる(図\ref{fig:snark_constraints_time}参照)。

したがって、本アーキテクチャは、以下のような特性を持つユースケースにおいて特に有効であると考えられる。
\begin{itemize}
    \item \textbf{クライアントサイドでの証明生成}: ユーザー自身のデバイスで証明を生成し、サーバーやブロックチェーンに送信するアプリケーション。
    例えば、プライバシーを保護した上での本人確認(分散型ID)、プライベートな状態を持つゲーム、機密データを用いた計算の検証などが挙げられる。
    \item \textbf{オンチェーンコストの最小化が重要}: ブロックチェーンのスケーラビリティが重視され、トランザクションコストを可能な限り抑えたいアプリケーション。
\end{itemize}

結論として、VOLEitHとSNARKを組み合わせたハイブリッドアプローチは、「証明者の高速性」と「検証者の低コスト」という、一見すると相反する要求を両立させるための有望なアーキテクチャである。
その性能は回路の特性、特に乗算の数に大きく依存するため、アプリケーションを設計する際には、このトレードオフを十分に理解することが重要となる。

\section{関連研究 (Related Work)}
本研究は、VOLEベースのゼロ知識証明をオンチェーンで実用化するための性能評価を行ったものである。本章では、他の主要なZKPシステム、VOLEベースZKPに関する先行研究、そしてオンチェーンZKP検証の取り組みと比較することで、本研究の位置付けを明確にする。

\subsection{他の主要なZKPシステムとの比較}
現在、ZKPシステムには多様なアプローチが存在し、それぞれが異なるトレードオフを持つ。
\begin{itemize}
    \item \textbf{zk-STARKs}: STARKsは、透明性(Trusted Setupが不要)とポスト量子耐性を大きな特徴とする。証明生成は高速だが、証明サイズがSNARKsと比較して大きく(数十〜数百KB)、オンチェーン検証のガス代が高くなる傾向がある。本研究で用いたVOLEitHもLPN仮定に基づくことでポスト量子耐性を持つが、証明圧縮のためにSNARK(Trusted Setupが必要)と組み合わせている点で、純粋なSTARKsとは異なるアプローチを取っている。
    \item \textbf{Plonk/Halo2}: これらのシステムは、Groth16のような特定の回路ごとにTrusted Setupを必要とするSNARKとは異なり、一度のTrusted Setup(Universal Trusted Setup)で様々な回路に再利用できるという利点を持つ。これにより開発の柔軟性が向上するが、一般的に証明サイズや検証コストはGroth16に比べて若干増加する。本研究では、検証コストの最小化を優先し、最も効率的なGroth16を圧縮に用いた。
    \item \textbf{R1CSベースのSNARKs (例: Circom/Groth16)}: 本研究でも比較対象とした、現在最も広く使われているZKPシステムの一つ。証明サイズと検証コストが非常に小さいという強力な利点を持つが、証明者の計算負荷が高く、複雑な回路では証明生成に長い時間を要する。本研究のアーキテクチャは、この証明者側の計算負荷をVOLEitHで軽減しようとする試みと位置付けられる。
\end{itemize}

\subsection{VOLEベースZKPに関する先行研究}
VOLEベースのZKPは、その証明者効率の高さから近年活発に研究されている分野である。	exttt{soft\_spoken}の元となった論文をはじめ、	exttt{mozzarella}、	exttt{mac-n-cheese}といった多くの実装が提案されている。これらの研究の多くは、プロトコルの理論的な改善や、オフラインでの性能評価に焦点を当てている。本研究の新規性は、これらのVOLEベースZKP実装の一つである	exttt{soft\_spoken}を実際に用い、SNARKによる圧縮を経てオンチェーンで検証するまでのエンドツーエンドのプロセスを構築し、その具体的な性能とコストを実測した点にある。

\subsection{オンチェーンZKP検証に関する既存の取り組み}
オンチェーンでのZKP検証は、ZKロールアップ(例: StarkNet, zkSync, Polygon zkEVM)によって大きな成功を収めている。これらのプロジェクトは、大量のトランザクションをオフチェーンで処理し、その正当性を単一のZKPでオンチェーン検証することで、ブロックチェーンのスケーラビリティを飛躍的に向上させた。これらの多くは、zk-STARKsやR1CSベースのSNARKsを利用している。

本研究のアプローチは、これらの大規模なロールアップとは異なり、個別のアプリケーションがクライアントサイドで生成した証明をオンチェーンで検証するような、より小規模で多様なユースケースを想定している。VOLEitHの高速な証明者性能は、Webブラウザやモバイルデバイスといったリソースが限られた環境でのZK機能の実装を可能にし、分散型IDやプライベート投票、オンチェーンゲームなど、新たなdAppsの領域を切り拓く可能性がある。本研究は、その実現に向けた第一歩となる具体的な性能データを提供するものである。

\section{結論と今後の展望 (Conclusion and Future Work)}

\subsection{結論}
本研究では、証明者効率の高いVOLE-in-the-Head(VOLEitH)プロトコルと、証明圧縮に優れたSNARKを組み合わせたハイブリッドアーキテクチャを構築し、そのオンチェーン検証における実現可能性と性能を定量的に評価した。

ベンチマークを通じて、以下の主要な知見が得られた。
\begin{enumerate}
    \item \textbf{VOLEitHの基本的なトレードオフ}: VOLEitHは、CircomのようなR1CSベースのシステムと比較して、証明者の計算を最大15.5倍高速化する一方で、生成される証明サイズが6000倍以上も増大することが確認された。この巨大な証明は、単体ではオンチェーン検証の大きな障壁となる。
    \item \textbf{SNARK圧縮の有効性}: VOLEitHの証明をSNARK(Groth16)で圧縮することにより、回路の複雑さに関わらず、最終的な証明サイズを1,055バイト、オンチェーン検証コストを約21万ガスに固定化できることを示した。これにより、VOLEitHのオンチェーン応用の道が拓かれる。
    \item \textbf{アーキテクチャのボトルネック}: エンドツーエンドのプロセスにおける主要なボトルネックは、SNARKの証明生成時間であり、これは回路内の乗算(ANDゲート)の数に起因する制約数に強く依存することが明らかになった。
\end{enumerate}
結論として、VOLEitHとSNARKを組み合わせたハイブリッドアプローチは、「クライアントサイドでの高速な証明生成」と「低コストなオンチェーン検証」という、一見すると相反する要求を両立しうる有望なアーキテクチャである。本研究は、その具体的な性能データとトレードオフを明らかにすることで、このアプローチの実用性に関する最初のマイルストーンを提示した。

\subsection{今後の展望}
本研究の成果を踏まえ、今後の展望として以下の方向性が考えられる。
\begin{itemize}
    \item \textbf{性能改善}:
    \begin{itemize}
        
        \item \textbf{SNARK証明生成の高速化}: 本アーキテクチャの主要なボトルネックであるSNARK証明生成時間を短縮するため、VOLEitHからR1CSへのより効率的な変換手法の研究や、Plonk/Halo2のような新しいSNARKシステムとの組み合わせを検討する必要がある。特に、Universal SNARKsを用いることで、開発の柔軟性向上も期待できる。
        
        \item \textbf{ガス代のさらなる削減}: 本研究では約21万ガスであった検証コストを、Verifierスマートコントラクトのさらなる最適化や、将来的なEVMのプリコンパイル追加などによって削減する研究が期待される。
    \end{itemize}
    \item \textbf{応用展開}:
    \begin{itemize}
        
        \item 本研究では基本的な回路を用いたが、今後は分散型ID(dID)、プライベートトランザクション、オンチェーンゲームのプライベート状態更新といった、より複雑で実用的なアプリケーションへの適用評価を進めるべきである。これにより、実際のユースケースにおける本アーキテクチャの有効性と課題がさらに明確になる。
    \end{itemize}
    \item \textbf{セキュリティの深化}:
    \begin{itemize}
        
        \item 本研究で採用したVOLEitHはLPN仮定に基づくポスト量子耐性を持つが、SNARK(Groth16)は量子コンピュータに対して脆弱である。エンドツーエンドでのポスト量子耐性を実現するため、STARKsのような耐量子性を持つ証明システムとの組み合わせを検討することは重要な研究課題である。また、アーキテクチャ全体としてのセキュリティレベルの形式的な分析も今後の課題となる。
    \end{itemize}
\end{itemize}


\newpage
\section*{参考文献 (References)}
\begin{thebibliography}{99}
\bibitem{eprint:iacr:2023:681}
Iden3.
Circom: A Circuit Compiler for Zero-Knowledge Proofs.
\textit{Cryptology ePrint Archive}, Paper 2023/681, 2023.
\url{https://eprint.iacr.org/2023/681}.
\end{thebibliography}

\end{document}
