\section{関連研究 (Related Work)}
VOLE-in-the-Head (VOLEitH)は\textbf{MPC-in-the-Head (MPCitH)}と\textbf{QuickSilver}という二つの先行研究パラダイムから多くの影響を受けている。
本節では、それぞれの研究系譜とVOLEitHとの接続点を整理し、本研究の貢献を位置付ける。

\subsection{MPC-in-the-Head (MPCitH) フレームワーク}
MPCitHは、VOLEitHが改良を目指したZKPパラダイムであり、セキュア多人数計算のアイデアを識別スキームやZKPoKに転用する枠組みである。

\subsubsection{起源と概念}
Ishai、Kushilevitz、Ostrovsky、Sahaiが2007年に提案したMPCitH(Multi-Party Computation in the Head)は、証明者 $P$ が秘密の証拠 $x$ を複数パーティのシェア $[x]_1,\ldots,[x]_N$ に分割し、仮想的なMPCプロトコルを頭の中で実行するというアプローチである。
検証者 $V$ は、ランダムに選んだパーティ $i^\*$ を除く全てのビューの開示を要求することで、証拠を知っていることを間接的に確認する。
任意の一方向関数 $F: x \mapsto y$ に対して適用できる汎用性があり、耐量子署名を含む幅広い暗号プロトコル構築に利用されてきた。

\subsubsection{耐量子署名への応用と限界}
MPCitHパラダイムは、耐量子署名スキームにおける重要な構成要素として、多数の候補を生み出している。
\begin{itemize}
    \item \textbf{Picnic} はMPCitHをベースにした代表的な署名スキームであり、NIST標準化候補として注目を集めた。
    \item \textbf{対称鍵プリミティブ}(AES、LowMC、Rainなど)をコアに据えたMPCitHスキームが提案され、実装容易性と効率のバランスを模索している。
    \item \textbf{追加PQC候補} として、AIMerやBiscuit、MIRA、MiRitH、MQOM、PERK、RYDE、SDitHなど、多数のMPCitH系署名が公募に提出された。
\end{itemize}
VOLEitHは、MPCitHに代わる手段として、よりシンプルかつ小さく高速な証明プロトコルを提供し、MPCitH系の証明者計算コストという限界を克服することを狙っている。

\subsubsection{VOLEitHが解決するMPCitHの課題}
MPCitHでは「多数の仮想パーティのうち一部のみを開示して検証する」という構成上、Soundnessエラーを抑えるためにチャレンジ\&回答の繰り返し回数を増やす必要がある。
Proof-of-Knowledge誤り $2^{-t}$ を達成するには$t$ラウンドのやり取りを必要とし、証明サイズは $O(t \cdot N)$ に膨らむ。
また証明者は各ラウンドで複数パーティ分のビューを再計算するため、線形演算と一方向関数評価がボトルネックになる。
VOLEitHは、QuickSilver由来のVOLE MACを使って乗算ゲート整合性を一括チェックすることで、MPCitHが抱えていた「多パーティ再現」「広帯域通信」といったオーバーヘッドを回路規模に線形なコストへ圧縮する。
さらに、SoftSpokenOTに代表されるOTリダクションを介して、MPCitHで必須だったビュー整合性証明をVOLEベースの一括生成処理に置き換え、Fiat--Shamir変換後でも短い証明と高速な証明者性能を維持できる。

\subsection{QuickSilver: VOLEベースZKPとの統合}
VOLEitHは、Vector Oblivious Linear Evaluation (VOLE) にアクセスできるQuickSilver型のZKPシステムを取り込み、VOLE MACを準同型コミットメントとして扱うことで効率を引き出す。

\subsubsection{QuickSilverの構造と特徴}
Yang、Sarkar、Weng、WangによるQuickSilverプロトコルは、VOLEハイブリッドモデルにおける対話型ZKPoKとして提案された。
DittmerらのLine-Point Zero-Knowledgeパラダイムを基盤とし、任意の有限体上の算術回路に対して証明を構成する。
VOLEから得られるタプル $(\mathbf{u}_i,\mathbf{v}_i,\mathbf{q}_i)$ は、グローバルキー $\mathbf{\Delta}$ の下で機能するメッセージ認証コードとなり、準同型性を活かして乗算制約 $\mathbf{w}_\alpha \cdot \mathbf{w}_\beta = \mathbf{w}_\gamma$ を効率的に検証する。
これにより、指定検証者型のVOLEベースZKPが、軽量な検証コストで高いスループットを実現できる。

\subsubsection{VOLEitH/FAESTへの取り込み}
FAEST署名では、QuickSilverスタイルのプロトコル $\Pi_{2D\text{-Rep}}$ をFOT-$\bar{1}$ハイブリッドモデルに設定し、O2Cコンパイラを適用して非対話型議論 $\Pi_{\text{FAEST}}$ を得る。
Fiat--Shamir変換を施すことでROM上の安全性が保証され、指定検証者向けだったVOLEプロトコルが公開検証可能なスキームへと昇華する。
VOLEitHは、SoftSpokenOTに代表されるOTベースの技術を活用したコンパイラにより、QuickSilver系プロトコルの利点を維持しつつ証明者を高速化し、我々のオンチェーン検証アーキテクチャに適した形で統合している。
todo: FAEST不要
todo: VOLE自体の説明が必要
