\section{関連研究 (Related Work)}
本研究は、VOLEベースのゼロ知識証明をオンチェーンで実用化するための性能評価を行ったものである。本章では、他の主要なZKPシステム、VOLEベースZKPに関する先行研究、そしてオンチェーンZKP検証の取り組みと比較することで、本研究の位置付けを明確にする。

\subsection{他の主要なZKPシステムとの比較}
現在、ZKPシステムには多様なアプローチが存在し、それぞれが異なるトレードオフを持つ。
\begin{itemize}
    \item \textbf{zk-STARKs}: STARKsは、透明性(Trusted Setupが不要)とポスト量子耐性を大きな特徴とする。証明生成は高速だが、証明サイズがSNARKsと比較して大きく(数十〜数百KB)、オンチェーン検証のガス代が高くなる傾向がある。本研究で用いたVOLEitHもLPN仮定に基づくことでポスト量子耐性を持つが、証明圧縮のためにSNARK(Trusted Setupが必要)と組み合わせている点で、純粋なSTARKsとは異なるアプローチを取っている。
    \item \textbf{Plonk/Halo2}: これらのシステムは、Groth16のような特定の回路ごとにTrusted Setupを必要とするSNARKとは異なり、一度のTrusted Setup(Universal Trusted Setup)で様々な回路に再利用できるという利点を持つ。これにより開発の柔軟性が向上するが、一般的に証明サイズや検証コストはGroth16に比べて若干増加する。本研究では、検証コストの最小化を優先し、最も効率的なGroth16を圧縮に用いた。
    \item \textbf{R1CSベースのSNARKs (例: Circom/Groth16)}: 本研究でも比較対象とした、現在最も広く使われているZKPシステムの一つ。証明サイズと検証コストが非常に小さいという強力な利点を持つが、証明者の計算負荷が高く、複雑な回路では証明生成に長い時間を要する。本研究のアーキテクチャは、この証明者側の計算負荷をVOLEitHで軽減しようとする試みと位置付けられる。
\end{itemize}

\subsection{VOLEベースZKPに関する先行研究}
VOLEベースのZKPは、その証明者効率の高さから近年活発に研究されている分野である。	exttt{soft\_spoken}の元となった論文をはじめ、	exttt{mozzarella}、	exttt{mac-n-cheese}といった多くの実装が提案されている。これらの研究の多くは、プロトコルの理論的な改善や、オフラインでの性能評価に焦点を当てている。本研究の新規性は、これらのVOLEベースZKP実装の一つである	exttt{soft\_spoken}を実際に用い、SNARKによる圧縮を経てオンチェーンで検証するまでのエンドツーエンドのプロセスを構築し、その具体的な性能とコストを実測した点にある。

\subsection{オンチェーンZKP検証に関する既存の取り組み}
オンチェーンでのZKP検証は、ZKロールアップ(例: StarkNet, zkSync, Polygon zkEVM)によって大きな成功を収めている。これらのプロジェクトは、大量のトランザクションをオフチェーンで処理し、その正当性を単一のZKPでオンチェーン検証することで、ブロックチェーンのスケーラビリティを飛躍的に向上させた。これらの多くは、zk-STARKsやR1CSベースのSNARKsを利用している。

本研究のアプローチは、これらの大規模なロールアップとは異なり、個別のアプリケーションがクライアントサイドで生成した証明をオンチェーンで検証するような、より小規模で多様なユースケースを想定している。VOLEitHの高速な証明者性能は、Webブラウザやモバイルデバイスといったリソースが限られた環境でのZK機能の実装を可能にし、分散型IDやプライベート投票、オンチェーンゲームなど、新たなdAppsの領域を切り拓く可能性がある。本研究は、その実現に向けた第一歩となる具体的な性能データを提供するものである。
