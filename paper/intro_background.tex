\section{序論 (Introduction)}
\subsection{ゼロ知識証明の進化とオンチェーン検証の課題}
ゼロ知識証明は、ある計算が正しく実行されたことを、その計算に関する入力情報を一切明らかにすることなく検証者に証明する暗号学的プロトコルである。
この性質はプライバシー保護が強く求められる現代のデジタル社会において極めて重要であり、ブロックチェーン技術の分野でもその応用が期待されている。
特にスマートコントラクトプラットフォームにおいては、計算の正当性をトランザクションとしてオンチェーンで検証する能力が、スケーラビリティとプライバシーの両方を向上させる鍵となる。
しかし、ZKPをオンチェーンで検証する際には、証明サイズ、検証計算量、そしてそれに伴うガス代という、ブロックチェーン特有の厳しい制約が大きな課題となる。

\subsection{VOLEベースZKPとVOLE-in-the-Head}
この課題に対し、証明者の計算効率を大幅に向上させる新しいZKPの系統として、VOLE(Vector Oblivious Linear Evaluation)ベースのプロトコルが登場した。
これらのプロトコルは、従来のSNARKsで主流であったR1CS(Rank-1 Constraint System)とは異なるアプローチを取り、特に証明者の計算負荷を軽減することに成功している。
TODO: SNARKの説明を追加

その中でもVOLE-in-the-Head (VOLE itH) は、VOLEベースの対話型プロトコルにFiat-Shamir変換を適用することで、誰でも検証可能な公開証明(publicly verifiable proof)を生成可能にした画期的な手法である。
これにより、証明者側の高い計算効率という利点を維持しつつ、オンチェーン検証への道が拓かれた。

\subsection{研究の目的と貢献}
VOLEitHは理論的には有望であるものの、その実用性、特にオンチェーン検証における具体的な性能やコストについては、まだ十分に明らかにされていない。
本研究の目的は、VOLEitHの特性を活かした軽量な証明者がEthereum上で検証することが可能であるか検証することにある。

具体的には、以下の項目を詳細に測定・分析する。
\begin{itemize}
    \item 証明生成と検証にかかる時間
    \item 生成される証明のサイズ
    \item 証明者と検証者の計算負荷(CPU、メモリ)
    \item 最終的なオンチェーン検証にかかるガス代
\end{itemize}
本研究は、VOLEitHのオンチェーン応用における実現可能性と技術的なトレードオフを明らかにすることで、将来的なZKPシステムの設計と最適化、そしてブロックチェーンアプリケーションにおけるプライバシーとスケーラビリティの向上に貢献することを目指す。

\section{実現可能性分析と主要な知見}
VOLEitHの証明をオンチェーンで扱うために複数の手法を検討し、最終的にGroth16によるSNARK Wrappingを実装・評価した。本節では、その意思決定過程と主要な洞察を整理する。

\subsection{SNARK Wrapping (Groth16)}
最も直接的なアプローチは、VOLEitHの検証ロジック全体をGroth16で実装方法である。この手法により、回路規模やゲート種別に依らず証明サイズを\textbf{1,055バイト}、オンチェーン検証ガスを\textbf{208,967 gas}に固定できた。一方で、制約数は
todo: snarkとgroth16,circomについて書く
\begin{align*}
16,640 \times n + 2,113,664
\end{align*}
と線形に増加し、特に非線形ゲートを多く含む回路では証明生成時間が急増する。実装は以下のコンポーネントで構成される。
\begin{itemize}
    \item \texttt{schmivitz-snark}: VOLEitH検証ロジックをarkworksベースのGroth16で証明するためのラッパー
    \item \texttt{VOLEitH-bench}: Groth16圧縮後の証明をEVMで検証し、エンドツーエンド性能を測定するベンチマーク
\end{itemize}
SHA-256のようにANDゲートが2万個を超える回路では、この制約爆発によりSNARKフェーズの実装を断念せざるを得なかった。
